% Created 2025-02-18 ter 15:38
% Intended LaTeX compiler: pdflatex
\documentclass[11pt]{article}
\usepackage[utf8]{inputenc}
\usepackage[T1]{fontenc}
\usepackage{graphicx}
\usepackage{longtable}
\usepackage{wrapfig}
\usepackage{rotating}
\usepackage[normalem]{ulem}
\usepackage{amsmath}
\usepackage{amssymb}
\usepackage{capt-of}
\usepackage{hyperref}
\author{Gustavo Magalhães Mendes de Tarso}
\date{2025-02-24}
\title{Análise de um Caso de Negociação e Gestão de Conflito}
\hypersetup{
 pdfauthor={Gustavo Magalhães Mendes de Tarso},
 pdftitle={Análise de um Caso de Negociação e Gestão de Conflito},
 pdfkeywords={},
 pdfsubject={},
 pdfcreator={Emacs 30.0.91 (Org mode 9.7.11)}, 
 pdflang={Pt-Br}}
\begin{document}

\maketitle
\section*{Introdução}
\label{sec:org622d705}
A negociação é um elemento essencial no ambiente corporativo, pois permite alinhar interesses divergentes e alcançar soluções eficazes. Em situações de conflito, a capacidade de ouvir diferentes perspectivas e manter a paciência são fatores críticos para o sucesso da negociação.

Este trabalho tem como objetivo analisar um caso real de negociação ocorrido no \textbf{Ministério da Previdência Social}, no qual foi necessário enfrentar um conflito com uma superior hierárquica a respeito de um cálculo matemático aplicado à distribuição de vagas do concurso de \textbf{Perícia Médica Federal}. Durante essa experiência, evidenciou-se a necessidade de aprimoramento em duas competências essenciais para um negociador eficaz: \textbf{escuta ativa e paciência}.
\section*{Características a Melhorar}
\label{sec:org16cb0f7}
\subsection*{Escuta Ativa}
\label{sec:orgc10f94d}
A \textbf{escuta ativa} é um componente fundamental da negociação, pois permite compreender completamente o ponto de vista da outra parte antes de formular uma resposta ou contraproposta. No caso analisado, houve dificuldade em ouvir a argumentação da superior hierárquica, dado que seu posicionamento era baseado em um erro matemático evidente. Essa dificuldade comprometeu a qualidade da comunicação e poderia ter gerado maior resistência ao convencimento.
\subsection*{Paciência}
\label{sec:org982d93b}
A paciência é uma habilidade essencial para a resolução de conflitos, pois permite conduzir discussões com equilíbrio e evitar confrontos desnecessários. No episódio relatado, a impaciência ao lidar com a resistência da chefe quase prejudicou o processo de convencimento, podendo ter levado a uma escalada do conflito.
\section*{Descrição do Caso}
\label{sec:orgc0798a6}
O caso ocorreu no contexto da elaboração da distribuição de vagas do concurso de \textbf{Perícia Médica Federal}, um processo que exigia precisão matemática para garantir justiça e eficiência na alocação dos recursos.

A chefe do setor possuía um perfil autoritário, utilizando sua posição hierárquica para impor decisões, mesmo em questões de natureza objetiva, como cálculos matemáticos. No entanto, ao revisar a metodologia proposta, identifiquei que o cálculo sugerido apresentava inconsistências que comprometeriam a exatidão da distribuição das vagas.

Ao expor os erros, houve resistência por parte da chefe, que insistiu na adoção de sua abordagem original, sem considerar os argumentos técnicos apresentados. O embate exigiu firmeza na defesa da alternativa correta, porém a dificuldade em escutar seu ponto de vista e a falta de paciência na condução do diálogo representaram desafios significativos ao longo da negociação.
\section*{Partes Envolvidas e seus Interesses}
\label{sec:orgc6778fd}
\begin{longtable}{l p{0.3\textwidth} p{0.3\textwidth}}
Parte & Interesse Principal & Objetivos\\
\hline
\endfirsthead
\multicolumn{3}{l}{Continued from previous page} \\
\hline

Parte & Interesse Principal & Objetivos \\

\hline
\endhead
\hline\multicolumn{3}{r}{Continued on next page} \\
\endfoot
\endlastfoot
\hline
\textbf{Chefe do setor} & Manutenção da autoridade decisória & Validar sua proposta e demonstrar controle sobre o processo\\
\hline
\textbf{Eu (servidor técnico)} & Precisão na distribuição das vagas & Garantir a correção matemática do cálculo para evitar distorções\\
\end{longtable}
\section*{Alternativas das Partes (MACNA)}
\label{sec:orgc1a6152}
A \textbf{Melhor Alternativa em Caso de Não Acordo (MACNA)} representa o que cada parte poderia fazer caso a negociação não chegasse a um consenso:

\begin{itemize}
\item \textbf{Chefe do setor:} Poderia ter imposto sua decisão de forma unilateral, utilizando sua autoridade para validar o cálculo incorreto, correndo o risco de gerar distorções no concurso e futuras contestações jurídicas.
\item \textbf{Eu:} Poderia ter escalado a questão para instâncias superiores ou utilizado relatórios técnicos para contestar a metodologia adotada. No entanto, essa alternativa poderia gerar atritos institucionais e comprometer o ambiente de trabalho.
\end{itemize}
\section*{Concessões Realizadas}
\label{sec:org7e2b876}
Dada a objetividade da questão matemática, \textbf{não houve concessões na solução final}, pois o cálculo correto foi adotado integralmente. No entanto, a condução da negociação poderia ter sido aprimorada por meio de uma abordagem mais estratégica, levando em conta a necessidade de preservar o relacionamento interpessoal e minimizar desgastes com a liderança.
\section*{Avaliação da Performance e Aprendizados}
\label{sec:org4ff9735}
O principal êxito da negociação foi garantir a correção do cálculo, protegendo a integridade do processo seletivo. No entanto, dois pontos críticos foram identificados para aprimoramento:

\begin{itemize}
\item \textbf{Escuta ativa:} Havia uma resistência em ouvir o argumento da chefe, mesmo que equivocado. Uma escuta mais atenta poderia ter facilitado a condução da negociação, permitindo identificar eventuais preocupações subjacentes dela.
\item \textbf{Paciência:} A impaciência na argumentação poderia ter gerado um impasse maior. O uso de técnicas de \textbf{negociação colaborativa}, como reformular perguntas e validar preocupações da outra parte, teria contribuído para um desfecho mais harmônico.
\end{itemize}
\section*{Plano de Desenvolvimento para Melhorar a Performance}
\label{sec:orgcad2ab7}
Para aprimorar as habilidades identificadas, algumas ações concretas serão implementadas em negociações futuras:

\begin{itemize}
\item \textbf{Treinar escuta ativa:} Aplicar técnicas como \textbf{paráfrase} e \textbf{validação emocional} para demonstrar que o ponto de vista do outro foi compreendido antes de apresentar contrapontos.
\item \textbf{Desenvolver paciência:} Utilizar abordagens baseadas em \textbf{perguntas abertas} e evitar respostas impulsivas, garantindo um diálogo mais estruturado e estratégico.
\item \textbf{Adotar técnicas de persuasão mais diplomáticas:} Utilizar dados e evidências de forma mais pedagógica, explicando o raciocínio de forma gradual e conduzindo a outra parte a chegar à conclusão correta por conta própria.
\end{itemize}
\section*{Conclusão}
\label{sec:orgeacf1e4}
O caso analisado demonstrou a importância da escuta ativa e da paciência em negociações organizacionais, especialmente quando há resistência hierárquica. Embora o resultado final tenha sido positivo, garantindo a adoção do cálculo correto, o processo poderia ter sido conduzido de forma mais estratégica para minimizar desgastes.

O desenvolvimento contínuo dessas habilidades será essencial para lidar com desafios futuros, garantindo que a comunicação em negociações seja cada vez mais eficaz e colaborativa.
\end{document}
