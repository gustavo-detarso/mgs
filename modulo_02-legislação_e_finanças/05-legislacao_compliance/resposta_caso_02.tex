% Created 2025-09-28 dom 23:23
% Intended LaTeX compiler: xelatex
\documentclass[12pt]{article}
\usepackage{graphicx}
\usepackage{longtable}
\usepackage{wrapfig}
\usepackage{rotating}
\usepackage[normalem]{ulem}
\usepackage{amsmath}
\usepackage{amssymb}
\usepackage{capt-of}
\usepackage{hyperref}
\usepackage{fontspec}
\setmainfont{TeX Gyre Termes}[
UprightFont=texgyretermes-regular.otf,
BoldFont=texgyretermes-bold.otf,
ItalicFont=texgyretermes-italic.otf,
BoldItalicFont=texgyretermes-bolditalic.otf
]
\usepackage[brazil, ]{babel}
\usepackage{microtype}
\usepackage[left=3cm,right=2cm,top=3cm,bottom=2cm]{geometry}
\setlength{\parindent}{0pt}
\setlength{\parskip}{6pt}
\sloppy
\usepackage{xcolor}
\usepackage{hyperref}
\hypersetup{colorlinks=true, linkcolor=black, urlcolor=blue, citecolor=black}
\newcommand{\blueunder}[2][0.5pt]{%
{\vspace{-1.2cm}\centering\color{blue}\rule{#2}{#1}\par}%
}
\usepackage{tabularray}
\UseTblrLibrary{booktabs}
\SetTblrInner{rowsep=2pt,colsep=6pt}
\renewcommand{\arraystretch}{1.12}
\usepackage[most]{tcolorbox}
\tcbset{boxrule=0.6pt, arc=2mm, colframe=black!50, colback=white, left=6pt, right=6pt, top=6pt, bottom=6pt}
\author{Gustavo M. Mendes de Tarso}
\date{\today}
\title{}
\hypersetup{
 pdfauthor={Gustavo M. Mendes de Tarso},
 pdftitle={},
 pdfkeywords={},
 pdfsubject={},
 pdfcreator={Emacs 28.2 (Org mode 9.5.5)}, 
 pdflang={Pt_Br}}
\begin{document}

\begin{center}
\includegraphics[height=2cm]{/home/gustavodetarso/Documentos/.share/mgs_org/fgv.png}
\end{center}
\blueunder{0.5\linewidth}

{\centering\color{blue}\itshape MBA Gestão em Saúde\par}

\begin{tcolorbox}[title=Dados do fichamento,
  colback=gray!5,colframe=gray!40,boxrule=0.4pt,sharp corners]
\begin{tblr}{Q[l,2.8cm] X[l]} % 2.8cm = largura da coluna de rótulos
\textbf{Autor(es)}       & Tarso, Gustavo. \\
\textbf{Título}          & Caso 2 da aula\\
\textbf{Módulo}          & Legislação e Finanças \\
\textbf{Disciplina}      & Legislação e Compliance \\
\textbf{Assuntos}        & SUS; Operadoras de Saúde; Ressarcimento \\
\textbf{Resumo}          & Ressarcimento das operadora de saúde para a União (SUS) \\
\end{tblr}
\end{tcolorbox}

\section*{Ressarcimento ao SUS pelas operadoras de planos de saúde}
\label{sec:org64f79d2}

\subsection*{Data da decisão dos Tribunais Superiores}
\label{sec:org96bbe39}
A obrigatoriedade do ressarcimento ao SUS foi consolidada pelo Supremo Tribunal Federal (STF) no julgamento do \textbf{\textbf{RE 597.064/RS}}, com repercussão geral reconhecida em \textbf{\textbf{2018}}. Nessa decisão, o STF manteve a constitucionalidade do art. 32 da Lei nº 9.656/1998, que obriga as operadoras a ressarcirem o SUS quando seus beneficiários utilizam a rede pública.

\subsection*{Fundamentos da decisão}
\label{sec:org578e6b0}
\begin{itemize}
\item O ressarcimento não viola o princípio da livre iniciativa nem o direito de propriedade, pois visa recompor os cofres públicos em função de um serviço prestado pelo SUS a beneficiários de planos privados.
\item O ressarcimento não configura tributo nem gera bitributação, sendo uma compensação legítima ao Estado.
\item O Estado não pode arcar com custos decorrentes de serviços que as operadoras privadas têm obrigação contratual de prestar.
\end{itemize}

\subsection*{Matérias de defesa possíveis pelas operadoras}
\label{sec:orgaae294c}
Dependendo do contrato, as operadoras podem invocar:
\begin{itemize}
\item \textbf{\textbf{Ausência de cobertura contratual:}} se o procedimento não constar no rol contratado pelo beneficiário.
\item \textbf{\textbf{Carência contratual vigente:}} quando o prazo de carência não tiver sido cumprido.
\item \textbf{\textbf{Não comunicação prévia:}} se o usuário não informou a operadora sobre a necessidade do atendimento.
\item \textbf{\textbf{Atendimento fora da área de cobertura geográfica:}} pode limitar a responsabilidade contratual.
\end{itemize}

---

\section*{Análise da decisão judicial sobre internação em UTI}
\label{sec:orgad4413d}

\subsection*{Fundamentação do paciente para a tutela provisória}
\label{sec:orgbebb9dc}
O autor, representado pela Defensoria Pública da União, fundamentou o pedido em:
\begin{itemize}
\item Quadro clínico gravíssimo: \textbf{\textbf{septissemia (CID A41)}}, \textbf{\textbf{pneumonia bacteriana (CID J15)}}, \textbf{\textbf{infecção urinária (CID N39)}} e \textbf{\textbf{insuficiência renal (CID N19)}}.
\item Risco iminente de morte e necessidade de transferência urgente para leito de UTI.
\item Inscrição na \textbf{\textbf{Central de Regulação de Leitos (CRESUS)}} com prioridade 1.
\end{itemize}

\subsection*{Princípios constitucionais invocáveis}
\label{sec:org918ea1b}
\begin{itemize}
\item \textbf{\textbf{Pelo Autor (paciente):}}  
\begin{itemize}
\item Direito à vida (art. 5º, caput, CF).
\item Direito à saúde (arts. 6º e 196, CF).
\item Dignidade da pessoa humana (art. 1º, III, CF).
\end{itemize}

\item \textbf{\textbf{Pelo Réu (Município):}}  
\begin{itemize}
\item Reserva do possível (limitações orçamentárias e estruturais).
\item Isonomia e impessoalidade (manutenção da ordem da fila e critérios técnicos iguais a todos).
\item Separação dos Poderes (respeito à gestão técnica e administrativa do Executivo).
\end{itemize}
\end{itemize}

\subsection*{Motivos invocados pelo julgador para indeferir a tutela}
\label{sec:orga072371}
\begin{itemize}
\item Reconhecimento do contexto excepcional da \textbf{\textbf{pandemia da COVID-19}}, declarada pela OMS em \textbf{\textbf{11 de março de 2020}}, com risco de \textbf{\textbf{colapso do sistema de saúde}}.
\item Necessidade de \textbf{\textbf{autocontenção do Judiciário}}, evitando intervenções que causem desorganização administrativa.
\item Possível \textbf{\textbf{efeito nocivo}} de autorizar a internação fora da ordem técnica e cronológica, prejudicando outros pacientes.
\item Confiança nas decisões dos gestores e profissionais de saúde, que devem seguir critérios clínicos internacionais para priorização de leitos.
\end{itemize}

---

\section*{Situações comuns de solicitações dos usuários do SUS}
\label{sec:org584d50f}

\begin{longtblr}[
  entry=none % sem legenda; remova se quiser caption
]{%
  width=\linewidth,
  colspec={X[1,l] X[1.8,l] X[1,l]}, % ajuste os pesos conforme o espaço
  rowhead=1                           % repete o cabeçalho nas páginas
}
\toprule
\textbf{Situação solicitada} &
\textbf{Possibilidade de atendimento} &
\textbf{Ente responsável} \\
\midrule
Fornecimento de medicamentos de alto custo não disponíveis &
Possível por ação judicial ou protocolo junto à Secretaria Estadual de Saúde &
Estado e União \\
Internação em leito de UTI por urgência/emergência &
Possível conforme critérios técnicos e disponibilidade de vagas &
Município (porta de entrada), Estado e União \\
Exames e consultas especializadas não disponíveis no município &
Possível via pactuação interfederativa (referenciamento) ou judicialização &
Município (encaminhamento) e Estado (oferta) \\
\bottomrule
\end{longtblr}

---

\section*{Conclusão}
\label{sec:orgafe51be}
A decisão analisada demonstra que, mesmo diante da urgência e do direito fundamental à saúde, o Poder Judiciário deve atuar com \textbf{\textbf{prudência e autocontenção}}, especialmente em contextos excepcionais como uma pandemia. O equilíbrio entre direitos individuais e coletivos, aliado ao respeito às decisões técnicas e administrativas dos gestores de saúde, fundamentou o indeferimento da internação em UTI no caso concreto.
\end{document}