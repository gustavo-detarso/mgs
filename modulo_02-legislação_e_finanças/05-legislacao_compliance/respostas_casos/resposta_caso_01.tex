% Created 2025-09-28 dom 23:49
% Intended LaTeX compiler: xelatex
\documentclass[12pt]{article}
\usepackage{graphicx}
\usepackage{longtable}
\usepackage{wrapfig}
\usepackage{rotating}
\usepackage[normalem]{ulem}
\usepackage{amsmath}
\usepackage{amssymb}
\usepackage{capt-of}
\usepackage{hyperref}
\usepackage{fontspec}
\setmainfont{TeX Gyre Termes}[
UprightFont=texgyretermes-regular.otf,
BoldFont=texgyretermes-bold.otf,
ItalicFont=texgyretermes-italic.otf,
BoldItalicFont=texgyretermes-bolditalic.otf
]
\usepackage[brazil, ]{babel}
\usepackage{microtype}
\usepackage[left=3cm,right=2cm,top=3cm,bottom=2cm]{geometry}
\setlength{\parindent}{0pt}
\setlength{\parskip}{6pt}
\sloppy
\usepackage{xcolor}
\usepackage{hyperref}
\hypersetup{colorlinks=true, linkcolor=black, urlcolor=blue, citecolor=black}
\newcommand{\blueunder}[2][0.5pt]{%
{\vspace{-1.2cm}\centering\color{blue}\rule{#2}{#1}\par}%
}
\usepackage{tabularray}
\UseTblrLibrary{booktabs}
\SetTblrInner{rowsep=2pt,colsep=6pt}
\renewcommand{\arraystretch}{1.12}
\usepackage[most]{tcolorbox}
\tcbset{boxrule=0.6pt, arc=2mm, colframe=black!50, colback=white, left=6pt, right=6pt, top=6pt, bottom=6pt}
\author{Gustavo M. Mendes de Tarso}
\date{\today}
\title{}
\hypersetup{
 pdfauthor={Gustavo M. Mendes de Tarso},
 pdftitle={},
 pdfkeywords={},
 pdfsubject={},
 pdfcreator={Emacs 28.2 (Org mode 9.5.5)}, 
 pdflang={Pt_Br}}
\begin{document}

\begin{center}
\includegraphics[height=2cm]{/home/gustavodetarso/Documentos/.share/mgs_org/fgv.png}
\end{center}
\blueunder{0.5\linewidth}

{\centering\color{blue}\itshape MBA Gestão em Saúde\par}

\begin{tcolorbox}[title=Dados do fichamento,
colback=gray!5,colframe=gray!40,boxrule=0.4pt,sharp corners]
\begin{tblr}{Q[l,2.8cm] X[l]} % 2.8cm = largura da coluna de rótulos
\textbf{Autor(es)}       & Tarso, Gustavo. \\
\textbf{Título}          & Case 1 da aula \\
\textbf{Módulo}          & Legislação e Finanças \\
\textbf{Disciplina}      & Legislação e Compliance \\
\textbf{Assuntos}        & Plataforma; consumidor.gov.br; Senacon \\
\textbf{Resumo}          & Apresentação do sistema consumidor.gov.br \\
\end{tblr}
\end{tcolorbox}

A plataforma \textbf{consumidor.gov.br} foi criada pela Secretaria Nacional do Consumidor (Senacon), vinculada ao Ministério da Justiça e Segurança Pública, com o objetivo de \textbf{aproximar consumidores e empresas, promovendo a solução direta, rápida, gratuita e transparente de conflitos de consumo por meio digital}. Trata-se de uma política pública que concretiza os princípios do \textbf{Código de Defesa do Consumidor (Lei nº 8.078/1990)}, ao oferecer um espaço oficial de diálogo e resolução consensual de demandas, reduzindo a necessidade de judicialização e fortalecendo a cidadania.

No setor de saúde, diversos segmentos empresariais podem participar da plataforma, desde que devidamente cadastrados e autorizados. Entre eles destacam-se as \textbf{operadoras de planos de saúde}, \textbf{empresas de seguro saúde e odontológico}, \textbf{laboratórios de análises clínicas e de diagnóstico por imagem}, \textbf{clínicas e redes de atendimento médico-hospitalar}, \textbf{empresas odontológicas}, \textbf{farmácias}, \textbf{drogarias} e \textbf{comércios eletrônicos de produtos médico-hospitalares e farmacêuticos}. Essa diversidade de participantes demonstra a amplitude da plataforma, que abrange toda a cadeia privada de prestação de serviços e fornecimento de produtos relacionados à saúde.

A participação empresarial na plataforma traz inúmeras vantagens estratégicas. Ao oferecer um canal direto de diálogo com os consumidores, as empresas podem \textbf{resolver conflitos de forma célere e menos onerosa}, evitando processos judiciais e sanções administrativas. Além disso, o histórico público de desempenho na plataforma — que exibe indicadores de resposta, tempo e taxa de solução — \textbf{fortalece a imagem institucional e a reputação da marca}, aumentando a confiança dos consumidores. Outro benefício é a possibilidade de \textbf{monitorar demandas recorrentes e identificar falhas nos processos internos}, o que contribui para a melhoria contínua dos serviços prestados.

Caso o consumidor não alcance a solução desejada por meio da plataforma, ele pode recorrer a outros instrumentos previstos na legislação, como os \textbf{Procons estaduais e municipais}, o \textbf{Juizado Especial Cível}, as \textbf{agências reguladoras competentes} (como a \textbf{Anvisa} e a \textbf{ANS}) ou as \textbf{ouvidorias das próprias empresas}. Esses meios podem ser utilizados \textbf{de forma concomitante ou gradativa}, de acordo com a natureza e a gravidade do conflito. Ainda assim, recomenda-se que o cidadão inicie pela plataforma consumidor.gov.br, por se tratar de um mecanismo \textbf{ágil, acessível e desburocratizado}, que frequentemente resulta em acordos satisfatórios sem necessidade de outras medidas.

Em síntese, a plataforma consumidor.gov.br representa um avanço significativo na política de defesa do consumidor no Brasil, especialmente no setor de saúde, ao \textbf{facilitar a resolução de conflitos, ampliar a transparência e estimular práticas empresariais mais responsáveis}. Para o cidadão, constitui uma ferramenta democrática de empoderamento e exercício de direitos; para as empresas, uma oportunidade de aprimorar a relação com seus clientes e fortalecer sua posição no mercado. Mesmo quando não resolve definitivamente a demanda, seu uso contribui para um sistema de consumo mais equilibrado, transparente e eficiente.
\end{document}