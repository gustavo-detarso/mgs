% Created 2025-11-20 qui 23:51
% Intended LaTeX compiler: lualatex
\documentclass[aspectratio=169]{beamer}
\usetheme{Madrid}
\usecolortheme{seagull}
\usepackage{graphicx}
\usepackage{longtable}
\usepackage{wrapfig}
\usepackage{rotating}
\usepackage[normalem]{ulem}
\usepackage{amsmath}
\usepackage{amssymb}
\usepackage{capt-of}
\usepackage{hyperref}
\usepackage{fontspec}
\setsansfont{TeX Gyre Heros}
\setmainfont{TeX Gyre Heros}
\usepackage{xcolor}
\definecolor{FGVBlue}{HTML}{003366}
\definecolor{FGVBlueLight}{HTML}{0070C0}
\definecolor{FGVGray}{HTML}{666666}
\usetheme{Madrid}
\setbeamercolor{palette primary}{bg=FGVBlue,fg=white}
\setbeamercolor{palette secondary}{bg=FGVBlueLight,fg=white}
\setbeamercolor{palette tertiary}{bg=FGVBlue,fg=white}
\setbeamercolor{palette quaternary}{bg=FGVBlue,fg=white}
\setbeamercolor{structure}{fg=FGVBlue}
\setbeamercolor{title}{fg=white,bg=FGVBlue}
\setbeamercolor{frametitle}{fg=white,bg=FGVBlue}
\setbeamercolor{normal text}{fg=FGVGray,bg=white}
\setbeamercolor{itemize item}{fg=FGVBlue}
\setbeamercolor{itemize subitem}{fg=FGVBlueLight}
\setbeamercolor{block title}{fg=white,bg=FGVBlue}
\setbeamercolor{block body}{fg=FGVGray,bg=FGVBlue!3}
\titlegraphic{\vspace{-0.5cm}\includegraphics[height=1.2cm]{/home/gustavodetarso/Documentos/.share/mgs_org/fgv.png}}
\AtBeginSection[]{}
\setbeamertemplate{navigation symbols}{}
\setbeamerfont{footline}{size=\scriptsize}
\setbeamertemplate{footline}{%
\leavevmode%
\hbox{%
\begin{beamercolorbox}[wd=.7\paperwidth,ht=2.5ex,dp=1ex,left]{author in head/foot}%
\hspace*{1ex}\insertshorttitle
\end{beamercolorbox}%
\begin{beamercolorbox}[wd=.3\paperwidth,ht=2.5ex,dp=1ex,right]{date in head/foot}%
\insertframenumber{} / \inserttotalframenumber\hspace*{2ex}
\end{beamercolorbox}%
}%
\vskip0pt%
}
\usetheme{default}
\author{Catarina Vila; Eduardo Blank; Francisco Robson L. de Moraes; Gustavo M. Mendes de Tarso; Milton Mello.}
\date{NOV/2026}
\title{Plano de Ação – Valores Universalizáveis em Saúde}
\hypersetup{
 pdfauthor={Catarina Vila; Eduardo Blank; Francisco Robson L. de Moraes; Gustavo M. Mendes de Tarso; Milton Mello.},
 pdftitle={Plano de Ação – Valores Universalizáveis em Saúde},
 pdfkeywords={},
 pdfsubject={},
 pdfcreator={Emacs 28.2 (Org mode 9.5.5)}, 
 pdflang={English}}
\begin{document}

\maketitle

\begin{frame}[label={sec:org6fbb867}]{Introdução}
Os valores universalizáveis — igualdade, equidade, respeito, honestidade e liberdade — são fundamentais para construir uma governança ética sólida e uma prática assistencial coerente.  
Este plano traduz esses valores em ações práticas estruturadas com base na metodologia \alert{5W2H}, indicadores \alert{SMART} e \alert{KPIs}, assegurando executabilidade, monitoramento contínuo e impacto institucional real.
\end{frame}

\begin{frame}[label={sec:org338c68c}]{Visão Geral do Fluxo do Plano}
\begin{center}
\includegraphics[width=0.135\textwidth]{fluxograma_valores_smart_kpi_colorido.png}
\end{center}
\end{frame}

\begin{frame}[label={sec:orge736bd5}]{Eixo 1 – Comitê de Valores Éticos}
Criado para fortalecer a governança institucional e padronizar decisões sobre dilemas éticos.  
O Comitê integra direção geral, jurídico, enfermagem, qualidade, RH e corpo clínico, garantindo diversidade de perspectivas e legitimidade organizacional.

\alert{\alert{Elementos-chave}}
\begin{itemize}
\item \alert{What:} criação e formalização do comitê
\item \alert{Why:} evitar decisões fragmentadas e aumentar a segurança ética
\item \alert{Where:} sede administrativa
\item \alert{When:} meses 1–2
\item \alert{Who:} direção, jurídico, qualidade, enfermagem, RH, corpo clínico
\item \alert{How:} reuniões quinzenais, atas padronizadas, fluxo formal de análise
\item \alert{How much:} \textasciitilde{}R\$ 3.000/ano
\item \alert{KPIs:} tempo de resposta < 10 dias; ≥ 70\% recomendações implementadas
\end{itemize}
\end{frame}

\begin{frame}[label={sec:org74ec732}]{Eixo 2 – Revisão do Código de Ética}
O Código de Ética atualizado incorpora valores universalizáveis, exemplos reais de dilemas e diretrizes práticas para enfrentar conflitos organizacionais.

\alert{\alert{Elementos-chave}}
\begin{itemize}
\item \alert{What:} revisão completa do Código de Ética e Conduta
\item \alert{Why:} reduzir ambiguidades normativas e fortalecer coerência moral
\item \alert{When:} meses 2–4
\item \alert{How:} oficinas, benchmark com boas práticas, diagramação acessível
\item \alert{How much:} \textasciitilde{}R\$ 5.000
\item \alert{KPIs:} 100\% de colaboradores comunicados; avaliação ≥ 80\%; revisão anual
\end{itemize}
\end{frame}

\begin{frame}[label={sec:orgcb742d3}]{Eixo 3 – Programa Ética no Cuidado}
Programa educacional contínuo que trabalha competências éticas, habilidades comunicacionais e análise sistemática de casos.

\alert{\alert{Elementos-chave}}
\begin{itemize}
\item \alert{What:} formação ética permanente (módulos digitais e presenciais)
\item \alert{Why:} decisões clínicas exigem preparo moral constante
\item \alert{When:} meses 4–12
\item \alert{How:} simulações, estudos de caso, oficinas interativas
\item \alert{How much:} \textasciitilde{}R\$ 25.000/ano
\item \alert{KPIs:} ≥ 85\% de adesão; pós-teste ≥ 80\%; nº de oficinas/ano
\end{itemize}
\end{frame}

\begin{frame}[label={sec:org49de052}]{Eixo 4 – Política de Equidade no Atendimento}
Estratégia institucional para reduzir vieses e desigualdades no acesso ao cuidado, assegurando justiça distributiva e respeito às diferenças.

\alert{\alert{Elementos-chave}}
\begin{itemize}
\item \alert{What:} revisão de fluxos de triagem, protocolos de priorização e materiais acessíveis
\item \alert{Why:} desigualdades sociais impactam diretamente a qualidade assistencial
\item \alert{Where:} pronto atendimento, ambulatório e internação
\item \alert{When:} a partir do mês 5
\item \alert{How much:} \textasciitilde{}R\$ 40.000
\item \alert{KPIs:} queixas de discriminação ↓ 20\%; diferença média de tempo de atendimento < 10\%
\end{itemize}
\end{frame}

\begin{frame}[label={sec:orgb24a581}]{Eixo 5 – Canal Ético e Espaço Seguro de Fala}
Canal anônimo que fortalece a integridade institucional e protege profissionais e pacientes contra silenciamento e retaliação.

\alert{\alert{Elementos-chave}}
\begin{itemize}
\item \alert{When:} meses 3–6
\item \alert{Who:} Compliance + TI
\item \alert{How:} plataforma independente, anonimato garantido, fluxo formal de apuração
\item \alert{How much:} \textasciitilde{}R\$ 18.000/ano
\item \alert{KPIs:} resposta < 7 dias; resolutividade > 85\%; satisfação dos denunciantes ≥ 80\%
\end{itemize}
\end{frame}

\begin{frame}[label={sec:org6174273}]{Eixo 6 – Protocolos de Decisão Ética}
Protocolos claros para tomada de decisão em situações críticas como:
\begin{itemize}
\item escassez de recursos
\item fim de vida
\item objeção de consciência
\item conflitos familiares
\end{itemize}

\alert{\alert{Elementos-chave}}
\begin{itemize}
\item \alert{When:} meses 6–12
\item \alert{Who:} Comitê de Bioética, Direção Médica, Jurídico
\item \alert{How:} construção multiprofissional, validação em casos reais, revisão periódica
\item \alert{How much:} \textasciitilde{}R\$ 10.000
\item \alert{KPIs:} adesão > 80\%; redução de conflitos e retrabalhos documentada
\end{itemize}
\end{frame}

\begin{frame}[label={sec:org650d544}]{Cronograma (Gantt)}
\begin{center}
\includegraphics[width=0.90\textwidth]{gantt-plano-valores.png}
\end{center}
\end{frame}

\begin{frame}[label={sec:orgfad307c}]{Conclusão}
O plano integra \alert{5W2H}, \alert{SMART} e \alert{KPIs} para transformar valores universalizáveis em práticas concretas, auditáveis e sustentáveis.  
O resultado esperado é um hospital com maior coerência moral, maior segurança decisória e maior maturidade ética, capaz de responder a dilemas complexos com estabilidade institucional e foco nos direitos dos pacientes e trabalhadores.
\end{frame}
\end{document}