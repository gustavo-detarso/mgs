% Created 2025-11-09 dom 18:18
% Intended LaTeX compiler: lualatex
\documentclass[12pt]{article}
\usepackage{graphicx}
\usepackage{longtable}
\usepackage{wrapfig}
\usepackage{rotating}
\usepackage[normalem]{ulem}
\usepackage{amsmath}
\usepackage{amssymb}
\usepackage{capt-of}
\usepackage{hyperref}
\usepackage{fontspec}
\defaultfontfeatures{Ligatures=TeX}
\setmainfont{Times New Roman}[
Path = /usr/share/fonts/truetype/msttcorefonts/,
Extension = .ttf,
UprightFont = Times_New_Roman,
BoldFont = Times_New_Roman_Bold,
ItalicFont = Times_New_Roman_Italic,
BoldItalicFont = Times_New_Roman_Bold_Italic
]
\usepackage[brazil, ]{babel}
\usepackage[a4paper,margin=3cm]{geometry}
\linespread{1.5}
\setlength{\parskip}{0.7em}
\setlength{\parindent}{0pt}
\usepackage{xcolor}
\usepackage{hyperref}
\hypersetup{colorlinks=true, linkcolor=black, urlcolor=blue, citecolor=black}
\newcommand{\blueunder}[2][0.5pt]{%
{\vspace{-1.2cm}\centering\color{blue}\rule{#2}{#1}\par}%
}
\usepackage{tabularray}
\UseTblrLibrary{booktabs}
\SetTblrInner{rowsep=1pt,colsep=4pt}
\renewcommand{\arraystretch}{1.0}
\usepackage[most]{tcolorbox}
\tcbset{boxrule=0.6pt, arc=2mm, colframe=black!50, colback=white, left=3pt, right=2pt, top=3pt, bottom=2pt}
\author{Gustavo M. Mendes de Tarso}
\date{\today}
\title{}
\hypersetup{
 pdfauthor={Gustavo M. Mendes de Tarso},
 pdftitle={},
 pdfkeywords={},
 pdfsubject={},
 pdfcreator={Emacs 28.2 (Org mode 9.5.5)}, 
 pdflang={Pt_Br}}
\begin{document}

\begin{center}
\includegraphics[height=2cm]{/home/gustavodetarso/Documentos/.share/mgs_org/fgv.png}
\end{center}
\blueunder{0.5\linewidth}

{\centering\color{blue}\itshape MBA Gestão em Saúde\par}

\begingroup
\linespread{1}\selectfont
\begin{tcolorbox}[title=Dados da obra,
colback=gray!5,colframe=gray!40,boxrule=0.4pt,sharp corners]
\begin{tblr}[rowsep=1pt,stretch=1]{rows={t}, colspec={Q[l,2.8cm] X[l]}}
\textbf{Título} & Na Rota do Dinheiro Sujo: Episódio “Remédio Amargo” \\
\textbf{Direção} & Alex Gibney \\
\textbf{Produção} & Jigsaw Productions / Netflix \\
\textbf{País de origem} & Estados Unidos \\
\textbf{Ano} & 2018 \\
\textbf{Duração} & 1h 3min \\
\textbf{Gênero} & Documentário / Investigativo \\
\textbf{Publicação} & Disponível em: Netflix \\
\textbf{Assuntos} & 1. Corrupção corporativa. 2. Indústria farmacêutica. 3. Governança e ética empresarial. 4. ESG. 5. Saúde pública. \\
\textbf{Resumo} & O episódio “Remédio Amargo” revela como a Valeant Pharmaceuticals, movida pela ganância, distorceu os princípios da boa governança e da responsabilidade social. A análise ESG evidencia o colapso ético de uma empresa que transformou a saúde em ativo especulativo. \\
\end{tblr}
\end{tcolorbox}
\endgroup
\linespread{1.5}\selectfont

ATIVIDADE 2 - Análise ESG: Caso "Remédio Amargo" (1,0 ponto)

Prazo: 21/11/2025, até 23:59h

Assistam ao episódio "Remédio Amargo" da primeira temporada da série "Na Rota do Dinheiro Sujo" (disponível na Netflix).

O que entregar:

\begin{itemize}
\item Um artigo analisando o caso apresentado sob a perspectiva dos aspectos ESG (Environmental, Social and Governance).
\end{itemize}

Requisitos:

\begin{itemize}
\item Extensão mínima: 1 página
\item A análise deve abordar claramente os três pilares do ESG: ambiental, social e governança
\item Relacione o caso com práticas e desafios do setor de saúde
\item Apresente uma visão crítica sobre as questões éticas e de governança retratadas
\end{itemize}

Entrega: e-class

\vspace{1cm}
\blueunder{0.5\linewidth}

“Remédio Amargo”: uma análise ESG do colapso ético na Valeant Pharmaceuticals

O episódio “Remédio Amargo”, da série Na Rota do Dinheiro Sujo (Netflix), apresenta o caso da Valeant Pharmaceuticals, uma companhia que ascendeu rapidamente por meio de aquisições agressivas, manipulação de preços e práticas financeiras duvidosas. A partir da perspectiva ESG (Environmental, Social and Governance), o documentário expõe de forma contundente como a ausência de responsabilidade ética e a obsessão pelo lucro imediato corroem os pilares da sustentabilidade corporativa, especialmente em um setor tão sensível quanto o da saúde.

Embora o foco do caso seja financeiro e ético, a dimensão ambiental se manifesta na negligência estrutural da empresa com práticas sustentáveis. O modelo de negócio da Valeant priorizava ganhos financeiros de curto prazo em detrimento de qualquer preocupação socioambiental. Não havia investimentos significativos em inovação verde, gestão de resíduos farmacêuticos ou controle de emissões — reflexo de uma governança voltada unicamente ao mercado financeiro e à valorização das ações.

O impacto social das ações da Valeant foi devastador. A empresa comprava medicamentos essenciais e elevava seus preços em até 800\%, tornando-os inacessíveis para milhares de pacientes. Essa prática violou frontalmente o princípio social do ESG, que exige que empresas de saúde garantam acesso, equidade e responsabilidade social. O documentário evidencia como a estratégia de maximização de lucros colocou o lucro acima da vida humana, afetando diretamente comunidades vulneráveis e gerando indignação pública. Essa conduta revela um rompimento ético com o propósito essencial da indústria da saúde: promover o bem-estar.

O pilar mais comprometido no caso Valeant é o da governança. A companhia foi estruturada sobre um modelo de negócios opaco, com operações financeiras artificiais e relacionamentos obscuros com farmácias intermediárias — o que configurou fraude contábil. A ausência de controles internos efetivos, a remuneração executiva baseada apenas em valorização de ações e a conivência do conselho de administração criaram um ambiente fértil para manipulação e escândalos. A governança da Valeant falhou em seu dever fiduciário, distorceu demonstrações financeiras e destruiu a confiança de investidores e consumidores.

O caso evidencia um dilema recorrente na indústria farmacêutica: a tensão entre inovação, ética e rentabilidade. Quando a governança se torna refém de metas financeiras e a dimensão social é ignorada, o próprio sistema de saúde perde legitimidade. O episódio mostra como práticas de precificação abusiva e fusões sem critérios técnicos desestruturam o equilíbrio entre acesso, pesquisa e sustentabilidade. O setor de saúde, portanto, precisa incorporar o ESG como eixo estratégico, garantindo que o lucro seja compatível com o valor social e ético da sua atuação.

O colapso da Valeant é um exemplo emblemático de como a ausência de governança ética e de responsabilidade social pode destruir valor econômico e reputacional. “Remédio Amargo” mostra que a sustentabilidade corporativa não se resume a relatórios de compliance, mas à coerência entre discurso e prática. A lição é clara: empresas de saúde que desconsideram os pilares ESG comprometem não apenas sua imagem, mas a própria credibilidade de um setor que deveria, antes de tudo, cuidar de pessoas.
\section*{Estatísticas do texto}
\label{sec:org9c172d0}
\begin{itemize}
\item Intervalo (Org): linhas 93 a 105 de 219
\item Linhas no intervalo (todas): 13
\item Linhas no intervalo (não vazias): 7
\item Palavras (no intervalo, texto limpo): 500
\item Parágrafos (no intervalo, texto limpo): 7
\item Linhas (PDF, FGV): 106
\end{itemize}
\end{document}