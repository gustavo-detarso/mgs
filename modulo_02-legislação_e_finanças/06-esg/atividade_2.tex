% Created 2025-11-09 dom 17:58
% Intended LaTeX compiler: lualatex
\documentclass[12pt]{article}
\usepackage{graphicx}
\usepackage{longtable}
\usepackage{wrapfig}
\usepackage{rotating}
\usepackage[normalem]{ulem}
\usepackage{amsmath}
\usepackage{amssymb}
\usepackage{capt-of}
\usepackage{hyperref}
\usepackage{fontspec}
\defaultfontfeatures{Ligatures=TeX}
\setmainfont{Times New Roman}[
Path = /usr/share/fonts/truetype/msttcorefonts/,
Extension = .ttf,
UprightFont = Times_New_Roman,
BoldFont = Times_New_Roman_Bold,
ItalicFont = Times_New_Roman_Italic,
BoldItalicFont = Times_New_Roman_Bold_Italic
]
\usepackage[brazil, ]{babel}
\usepackage[a4paper,margin=3cm]{geometry}
\linespread{1.5}
\setlength{\parskip}{0.7em}
\setlength{\parindent}{0pt}
\usepackage{xcolor}
\usepackage{hyperref}
\hypersetup{colorlinks=true, linkcolor=black, urlcolor=blue, citecolor=black}
\newcommand{\blueunder}[2][0.5pt]{%
{\vspace{-1.2cm}\centering\color{blue}\rule{#2}{#1}\par}%
}
\usepackage{tabularray}
\UseTblrLibrary{booktabs}
\SetTblrInner{rowsep=1pt,colsep=4pt}
\renewcommand{\arraystretch}{1.0}
\usepackage[most]{tcolorbox}
\tcbset{boxrule=0.6pt, arc=2mm, colframe=black!50, colback=white, left=3pt, right=2pt, top=3pt, bottom=2pt}
\author{Gustavo M. Mendes de Tarso}
\date{\today}
\title{}
\hypersetup{
 pdfauthor={Gustavo M. Mendes de Tarso},
 pdftitle={},
 pdfkeywords={},
 pdfsubject={},
 pdfcreator={Emacs 28.2 (Org mode 9.5.5)}, 
 pdflang={Pt_Br}}
\begin{document}

\begin{center}
\includegraphics[height=2cm]{/home/gustavodetarso/Documentos/.share/mgs_org/fgv.png}
\end{center}
\blueunder{0.5\linewidth}

{\centering\color{blue}\itshape MBA Gestão em Saúde\par}

\begingroup
\linespread{1}\selectfont
\begin{tcolorbox}[title=Dados da obra,
colback=gray!5,colframe=gray!40,boxrule=0.4pt,sharp corners]
\begin{tblr}[rowsep=1pt,stretch=1]{rows={t}, colspec={Q[l,2.8cm] X[l]}}
\textbf{Título} & Na Rota do Dinheiro Sujo: Episódio “Remédio Amargo” \\
\textbf{Direção} & Alex Gibney \\
\textbf{Produção} & Jigsaw Productions / Netflix \\
\textbf{País de origem} & Estados Unidos \\
\textbf{Ano} & 2018 \\
\textbf{Duração} & 1h 3min \\
\textbf{Gênero} & Documentário / Investigativo \\
\textbf{Publicação} & Disponível em: Netflix \\
\textbf{Assuntos} & 1. Corrupção corporativa. 2. Indústria farmacêutica. 3. Governança e ética empresarial. 4. ESG. 5. Saúde pública. \\
\textbf{Resumo}      & O episódio “Remédio Amargo” expõe como a busca por lucro no setor farmacêutico levou à crise dos opioides, revelando graves falhas éticas e de governança. A análise ESG mostra o colapso dos pilares ambiental, social e de governança diante da negligência corporativa e da manipulação do sistema de saúde. \\
\end{tblr}
\end{tcolorbox}
\endgroup
\linespread{1.5}\selectfont

ATIVIDADE 2 - Análise ESG: Caso "Remédio Amargo" (1,0 ponto)
Prazo: 21/11/2025, até 23:59h

Assistam ao episódio "Remédio Amargo" da primeira temporada da série "Na Rota do Dinheiro Sujo" (disponível na Netflix).

O que entregar:

\begin{itemize}
\item Um artigo analisando o caso apresentado sob a perspectiva dos aspectos ESG (Environmental, Social and Governance).
\end{itemize}

Requisitos:

\begin{itemize}
\item Extensão mínima: 1 página
\item A análise deve abordar claramente os três pilares do ESG: ambiental, social e governança
\item Relacione o caso com práticas e desafios do setor de saúde
\item Apresente uma visão crítica sobre as questões éticas e de governança retratadas
\end{itemize}

Entrega: e-class

\vspace{1cm}
\blueunder{0.5\linewidth}

Título: “Remédio Amargo”: uma análise ESG sobre corrupção e responsabilidade corporativa no setor farmacêutico

O episódio “Remédio Amargo”, da série Na Rota do Dinheiro Sujo (Netflix), revela as práticas escusas de grandes corporações farmacêuticas nos Estados Unidos e os impactos devastadores da busca incessante por lucro sobre a sociedade. A partir da perspectiva ESG — Environmental, Social and Governance — o caso retrata com clareza como a negligência ética e a fragilidade dos mecanismos de controle podem gerar crises humanitárias, ambientais e institucionais de grande magnitude.

Embora o foco central do episódio seja a manipulação de medicamentos e a corrupção corporativa, há implicações ambientais importantes. A produção em larga escala de opioides e o descarte inadequado de substâncias químicas têm reflexos diretos sobre o meio ambiente, especialmente no manejo de resíduos farmacêuticos e na contaminação de águas e solos. O caso evidencia a ausência de uma política ambiental integrada às práticas de responsabilidade corporativa: as empresas priorizam o lucro e o controle de mercado, ignorando os impactos indiretos de sua cadeia produtiva. Uma agenda ESG madura exigiria dessas corporações transparência ambiental, rastreabilidade dos resíduos e investimentos em biotecnologia limpa.

O pilar social é o mais evidente no episódio. As empresas farmacêuticas mostradas manipulam médicos, pacientes e instituições públicas para ampliar o consumo de opioides, promovendo uma epidemia que resultou em milhares de mortes e famílias destruídas. A dimensão social do ESG é frontalmente violada: o direito à saúde e à informação é substituído pela exploração da vulnerabilidade humana. Essa prática rompe o contrato social básico entre a indústria da saúde e a sociedade — o de promover o bem-estar coletivo. Além disso, o episódio expõe como as corporações se valem de estratégias de marketing agressivas e bonificações a médicos, distorcendo o propósito terapêutico do medicamento e substituindo a ética por incentivos financeiros.

A crise exposta em “Remédio Amargo” é, sobretudo, uma crise de governança. A ausência de controles internos efetivos, a conivência de conselhos administrativos e a captura regulatória de órgãos públicos demonstram falhas profundas nos mecanismos de compliance. A governança corporativa das empresas retratadas falha em assegurar princípios básicos como integridade, transparência e responsabilidade. O caso ilustra o que ocorre quando o compliance se torna apenas formal, sem o compromisso real da alta liderança. A governança ética, em sentido amplo, deveria incluir canais de denúncia protegidos, auditorias independentes e alinhamento entre remuneração executiva e critérios de impacto social positivo.

O episódio oferece um espelho para o setor de saúde como um todo, inclusive fora dos Estados Unidos. Mostra como a dependência de incentivos financeiros e a pressão por resultados podem corroer a ética médica e administrativa. No contexto de políticas públicas, o caso reforça a necessidade de regulação forte, fiscalização técnica independente e mecanismos de responsabilização civil e criminal para corporações e gestores que se desviam de seu dever de cuidado. A sustentabilidade do setor de saúde depende de decisões que equilibrem inovação e segurança, lucro e equidade, crescimento e responsabilidade social.

“Remédio Amargo” não é apenas um relato de ganância; é um estudo de caso sobre a falência dos princípios ESG quando são tratados como discurso e não como prática. O episódio revela que o verdadeiro remédio para o sistema de saúde — público ou privado — está na integridade das suas instituições e no compromisso das lideranças com valores éticos, humanos e ambientais. A lição central é clara: sem governança responsável e visão social, o lucro rápido se transforma em dano coletivo duradouro.
\section*{Estatísticas do texto}
\label{sec:org33d6da3}
\begin{itemize}
\item Intervalo (Org): linhas 93 a 105 de 217
\item Linhas no intervalo (todas): 13
\item Linhas no intervalo (não vazias): 7
\item Palavras (no intervalo, texto limpo): 562
\item Parágrafos (no intervalo, texto limpo): 6
\item Linhas (PDF, FGV): 113
\end{itemize}
\end{document}