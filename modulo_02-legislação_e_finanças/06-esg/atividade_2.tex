% Created 2025-11-20 qui 23:48
% Intended LaTeX compiler: lualatex
\documentclass[12pt]{article}
\usepackage{graphicx}
\usepackage{longtable}
\usepackage{wrapfig}
\usepackage{rotating}
\usepackage[normalem]{ulem}
\usepackage{amsmath}
\usepackage{amssymb}
\usepackage{capt-of}
\usepackage{hyperref}
\usepackage{fontspec}
\defaultfontfeatures{Ligatures=TeX}
\setmainfont{Times New Roman}[
Path = /usr/share/fonts/truetype/msttcorefonts/,
Extension = .ttf,
UprightFont = Times_New_Roman,
BoldFont = Times_New_Roman_Bold,
ItalicFont = Times_New_Roman_Italic,
BoldItalicFont = Times_New_Roman_Bold_Italic
]
\usepackage[brazil, ]{babel}
\usepackage[a4paper,margin=3cm]{geometry}
\linespread{1.5}
\setlength{\parskip}{0.7em}
\setlength{\parindent}{0pt}
\usepackage{xcolor}
\usepackage{hyperref}
\hypersetup{colorlinks=true, linkcolor=black, urlcolor=blue, citecolor=black}
\newcommand{\blueunder}[2][0.5pt]{%
{\vspace{-1.2cm}\centering\color{blue}\rule{#2}{#1}\par}%
}
\usepackage{tabularray}
\UseTblrLibrary{booktabs}
\SetTblrInner{rowsep=1pt,colsep=4pt}
\renewcommand{\arraystretch}{1.0}
\usepackage[most]{tcolorbox}
\tcbset{boxrule=0.6pt, arc=2mm, colframe=black!50, colback=white, left=3pt, right=2pt, top=3pt, bottom=2pt}
\author{Gustavo M. Mendes de Tarso}
\date{\today}
\title{}
\hypersetup{
 pdfauthor={Gustavo M. Mendes de Tarso},
 pdftitle={},
 pdfkeywords={},
 pdfsubject={},
 pdfcreator={Emacs 28.2 (Org mode 9.5.5)}, 
 pdflang={Pt_Br}}
\begin{document}

\begin{center}
\includegraphics[height=2cm]{/home/gustavodetarso/Documentos/.share/mgs_org/fgv.png}
\end{center}
\blueunder{0.5\linewidth}

{\centering\color{blue}\itshape MBA Gestão em Saúde\par}

\begingroup
\linespread{1}\selectfont
\begin{tcolorbox}[title=Dados da obra,
colback=gray!5,colframe=gray!40,boxrule=0.4pt,sharp corners]
\begin{tblr}[rowsep=1pt,stretch=1]{rows={t}, colspec={Q[l,2.8cm] X[l]}}
\textbf{Título} & Na Rota do Dinheiro Sujo: Episódio “Remédio Amargo” \\
\textbf{Direção} & Alex Gibney \\
\textbf{Produção} & Jigsaw Productions / Netflix \\
\textbf{País de origem} & Estados Unidos \\
\textbf{Ano} & 2018 \\
\textbf{Duração} & 1h 3min \\
\textbf{Gênero} & Documentário / Investigativo \\
\textbf{Publicação} & Disponível em: Netflix \\
\textbf{Assuntos} & 1. Corrupção corporativa. 2. Indústria farmacêutica. 3. Governança e ética empresarial. 4. ESG. 5. Saúde pública. \\
\textbf{Resumo}      & O episódio “Remédio Amargo” expõe como a busca por lucro no setor farmacêutico levou à crise dos opioides, revelando graves falhas éticas e de governança. A análise ESG mostra o colapso dos pilares ambiental, social e de governança diante da negligência corporativa e da manipulação do sistema de saúde. \\
\end{tblr}
\end{tcolorbox}
\endgroup
\linespread{1.5}\selectfont

\vspace{1cm}
\blueunder{0.5\linewidth}

O episódio “Remédio Amargo”, da série Na Rota do Dinheiro Sujo, traz a história da Valeant Pharmaceuticals de um jeito que incomoda — e deveria incomodar. O documentário mostra como uma empresa que deveria existir para melhorar vidas acabou se tornando símbolo de ganância, manipulação financeira e indiferença humana. Ao olhar esse caso pela lente ESG, é impossível não enxergar o quanto a busca cega pelo lucro, quando descolada de qualquer responsabilidade ética, pode transformar o setor da saúde em algo profundamente desumano.

Apesar de o foco do episódio ser o comportamento financeiro e contábil da empresa, fica claro que a dimensão ambiental quase nem existia dentro da Valeant. A companhia não parecia minimamente preocupada com práticas sustentáveis, inovação responsável ou impacto ambiental. A lógica era simples e brutal: cortar custos, comprar empresas, aumentar preços, inflar o valor das ações. Num modelo que depende de resultados imediatos e recompensas executivas atreladas ao preço da ação, pensar em sustentabilidade — seja ambiental ou social — virava quase um obstáculo.

Mas o que realmente escancara a profundidade do problema é o impacto social. A Valeant comprava medicamentos essenciais e multiplicava seus preços por cinco, por oito, por dez — sem qualquer justificativa clínica, sem melhora terapêutica, sem pesquisa adicional. E do outro lado desses preços estavam famílias reais, muitas de classe média baixa, que de um mês para o outro não conseguiam mais pagar um remédio que mantinha um filho vivo, que controlava uma doença crônica, que permitia que alguém continuasse trabalhando. O documentário mostra esse sofrimento de forma dura: pessoas sendo obrigadas a escolher entre pagar o aluguel ou comprar um medicamento que sempre esteve ao alcance. É devastador.

E essa dor social tem tudo a ver com a estrutura do sistema de saúde dos Estados Unidos. Lá, a maior parte das pessoas depende de operadoras de saúde privadas (health insurance companies). Esses planos funcionam numa lógica mercadológica: eles negociam preços com as farmacêuticas, definem quais remédios serão cobertos e estabelecem quanto o paciente terá de pagar do próprio bolso. Quando a Valeant disparou artificialmente o preço dos medicamentos, muitos planos simplesmente deixaram de cobrir ou passaram a exigir co-pagamentos absurdos. Assim, o efeito não ficou restrito à indústria — caiu diretamente no colo de quem menos tinha poder para reagir: famílias comuns, pacientes com doenças raras, idosos que não conseguem trabalhar para complementar renda. A engrenagem inteira foi contaminada por essa lógica predatória.

O episódio também traz um elemento pouco comentado no debate público brasileiro, mas central no caso: a atuação dos investidores que faziam venda a descoberto (short selling). Isso significa operar no mercado apostando que o preço da ação vai cair — na prática, o investidor pega ações emprestadas, vende por um preço alto e depois recompra quando (ou se) o preço cair. É uma aposta contra a empresa. No caso da Valeant, esses investidores soaram o alarme muito antes do escândalo estourar: eles identificaram inconsistências, estruturas financeiras suspeitas e alertaram que “algo cheirava mal”. Foram muito atacados na época, mas acabaram se mostrando corretos. Paradoxalmente, foram eles — investidores especuladores — que ajudaram a desmontar a narrativa de uma empresa que se vendia como inovadora, mas que na verdade construía valor com alicerces fraudulentos.

O pilar da governança, nesse caso, não apenas falhou: ele ruiu por completo. A Valeant operava com engenharia financeira opaca, usava farmácias intermediárias para inflar números e apresentava resultados distorcidos para o mercado. A remuneração dos executivos era inteiramente guiada pelo preço da ação, criando um incentivo direto para práticas de risco e manipulação. Era um ambiente onde conselhos de administração preferiam não perguntar, auditores não aprofundavam e acionistas celebravam enquanto o castelo de cartas crescia.

Para mim, o caso da Valeant evidencia algo que vai além dessa empresa específica: expõe o limite ético do capitalismo. Sei que vivemos dentro desse sistema e que não temos, no curto prazo, uma alternativa funcional para substituí-lo. Mas também sei que é ilusório acreditar que o lucro, por si só, pode ser um princípio ético. O capitalismo não recompensa cuidado, responsabilidade ou empatia — recompensa retorno financeiro. E é exatamente por isso que, se estamos condenados a conviver com esse modelo, o mínimo que se espera é que ele seja conduzido com ética, transparência e respeito humano. Sem isso, o resultado é sempre o mesmo: sofrimento real de pessoas reais.

O colapso da Valeant não foi apenas o colapso de uma empresa; foi o colapso de uma lógica que prioriza o valor de mercado acima da vida. “Remédio Amargo” não fala só sobre uma fraude corporativa — fala sobre o risco social de permitir que a saúde seja tratada apenas como commodity. E a lição é clara: quando uma empresa de saúde ignora ESG, ela não está apenas destruindo sua reputação; está destruindo o que deveria ser sua razão de existir — o cuidado com pessoas.
\end{document}