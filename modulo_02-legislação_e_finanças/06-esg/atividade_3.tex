% Created 2025-11-20 qui 23:28
% Intended LaTeX compiler: lualatex
\documentclass[12pt]{article}
\usepackage{graphicx}
\usepackage{longtable}
\usepackage{wrapfig}
\usepackage{rotating}
\usepackage[normalem]{ulem}
\usepackage{amsmath}
\usepackage{amssymb}
\usepackage{capt-of}
\usepackage{hyperref}
\usepackage{fontspec}
\defaultfontfeatures{Ligatures=TeX}
\setmainfont{Times New Roman}[
Path = /usr/share/fonts/truetype/msttcorefonts/,
Extension = .ttf,
UprightFont = Times_New_Roman,
BoldFont = Times_New_Roman_Bold,
ItalicFont = Times_New_Roman_Italic,
BoldItalicFont = Times_New_Roman_Bold_Italic
]
\usepackage[brazil, ]{babel}
\usepackage[a4paper,margin=3cm]{geometry}
\linespread{1.5}
\setlength{\parskip}{0.7em}
\setlength{\parindent}{0pt}
\usepackage{xcolor}
\usepackage{hyperref}
\hypersetup{colorlinks=true, linkcolor=black, urlcolor=blue, citecolor=black}
\newcommand{\blueunder}[2][0.5pt]{%
{\vspace{-1.2cm}\centering\color{blue}\rule{#2}{#1}\par}%
}
\usepackage{tabularray}
\UseTblrLibrary{booktabs}
\SetTblrInner{rowsep=1pt,colsep=4pt}
\renewcommand{\arraystretch}{1.0}
\usepackage[most]{tcolorbox}
\tcbset{boxrule=0.6pt, arc=2mm, colframe=black!50, colback=white, left=3pt, right=2pt, top=3pt, bottom=2pt}
\author{Gustavo M. Mendes de Tarso}
\date{\today}
\title{}
\hypersetup{
 pdfauthor={Gustavo M. Mendes de Tarso},
 pdftitle={},
 pdfkeywords={},
 pdfsubject={},
 pdfcreator={Emacs 28.2 (Org mode 9.5.5)}, 
 pdflang={Pt_Br}}
\begin{document}

\begin{center}
\includegraphics[height=2cm]{/home/gustavodetarso/Documentos/.share/mgs_org/fgv.png}
\end{center}
\blueunder{0.5\linewidth}

\begingroup
\linespread{1}\selectfont
\begin{tcolorbox}[title=Ficha Técnica,
  colback=gray!5,colframe=gray!40,boxrule=0.4pt,sharp corners]
\begin{tblr}[rowsep=1pt,stretch=1]{rows={t}, colspec={Q[l,2.8cm] X[l]}}
\textbf{Curso}   		& MBA em Gestão da Saúde \\
\textbf{Turma}      		& T-01 \\
\textbf{Pólo}      		& Brasília \\
\textbf{Disciplina}     	& Governança Corporativa e Compliance \\
\textbf{Professor}      	& Eduardo Rosa Pedreira \\
\textbf{Trabalho}  	        & ATIVIDADE 3 - Plano de Ação: Valores Universalizáveis em Saúde (8,00 pontos) \\
\textbf{Aluno(s)}     		& 1. Catarina Vila; 2. Eduardo Blank; 3. Francisco Robson L. de Moraes; 4. Gustavo M. Mendes de Tarso; 5. Milton Mello. \\
\textbf{Prazo}     		& 21 de novembro de 2025 \\
\textbf{Título do trabalho}     & Plano de Ação para Implementação de Valores Universalizáveis em um Hospital Brasileiro \\ 
\end{tblr}
\end{tcolorbox}
\endgroup

\vspace{1cm}
\blueunder{0.5\linewidth}

\linespread{1.5}\selectfont

A incorporação de valores universalizáveis — igualdade, equidade, respeito, honestidade e liberdade — exige que a instituição adote uma lógica de ação concreta, planejada e monitorável, capaz de transformar princípios éticos em práticas operacionais consistentes. Para isso, cada iniciativa deste plano é construída segundo o método 5W2H, que permite explicitar o que será realizado (What), a razão de sua necessidade (Why), o local de execução (Where), o período de implementação (When), os responsáveis diretos (Who), o modo como será conduzida (How) e os custos envolvidos (How much). Esses elementos se entrelaçam à lógica SMART (específico, mensurável, alcançável, relevante e temporal), que transforma cada ação em meta verificável, e aos KPIs que permitem medir o grau real de avanço da cultura ética ao longo do tempo. Assim, em vez de ideias abstratas, o plano se materializa como um conjunto articulado de ações operacionais, com objetivos claros, métricas definidas e monitoramento contínuo, garantindo que os valores universalizáveis permeiem processos administrativos, decisões clínicas e relações humanas em toda a instituição.

A criação do Comitê de Valores Éticos e Sustentabilidade é a primeira ação estruturante, porque define um centro de gravidade capaz de irradiar coerência ética a todas as demais iniciativas. O What consiste em formalizar, por meio de portaria, um colegiado institucional que funcionará como guardião dos valores e mediador de dilemas complexos. O Why deixa claro que, sem uma instância institucionalizada, os valores tendem a se diluir diante da sobrecarga assistencial ou da pressão por decisões rápidas, e os conflitos tendem a ser tratados de forma informal e desigual. O Where, localizado na sede administrativa, reforça o caráter institucional do comitê, assegurando acesso aos documentos oficiais, às equipes estratégicas e à governança. O When — entre o primeiro e o segundo mês — insere o comitê como alicerce inicial do plano, garantindo que as demais ações tenham respaldo técnico e ético desde o início. O Who inclui direção geral, jurídico, enfermagem, corpo clínico, qualidade e RH, compondo uma instância multiprofissional capaz de analisar dilemas sob diferentes perspectivas. O How se traduz em reuniões quinzenais, análise sistemática de casos, produção de atas e fluxos padronizados de deliberação. O How much é relativamente baixo — cerca de R\$ 3 mil anuais — o que demonstra a viabilidade financeira da ação. Em termos SMART, o comitê é específico (tem missão clara), mensurável (número de reuniões, tempo de resposta e percentual de recomendações implementadas), alcançável (estrutura leve), relevante (impacto direto sobre a cultura) e temporal (implantação no mês 1–2). Seus KPIs — como meta de 24 reuniões anuais, tempo médio de resposta inferior a 10 dias úteis e implementação acima de 70\% das recomendações — não são apenas métricas administrativas, mas indicadores da vitalidade ética da organização, revelando se o comitê se consolidou como instância de referência ou se permanece apenas formalizado no papel.

A revisão do Código de Ética e Conduta representa o segundo eixo e, embora pareça um documento normativo simples, atua como ferramenta estratégica de alinhamento moral e organizacional. O What consiste em reescrever o documento institucional para incorporar valores universalizáveis, exemplos reais de dilemas e orientações aplicáveis ao cotidiano. O Why evidencia que códigos genéricos, comuns em hospitais pequenos, não dão conta de orientar condutas em situações de conflito porque deixam margens interpretativas amplas, produzindo insegurança jurídica e moral. O Where — o setor de Qualidade e Governança Clínica — é estratégico porque reúne especialistas na produção e revisão de documentos institucionais. O When — meses dois a quatro — posiciona a revisão logo após a criação do comitê, permitindo que o novo código dialogue com as deliberações iniciais e com o mapeamento de maiores necessidades. O Who inclui o Comitê de Valores Éticos, o Jurídico e profissionais de qualidade, numa construção coletiva que garante robustez técnica e legitimidade. O How envolve análise comparada de códigos nacionais e internacionais, oficinas de revisão e processos de validação com diversas equipes. O How much — cerca de R\$ 5 mil — cobre diagramação, impressão digital e ações de comunicação. Pelo prisma SMART, trata-se de uma ação específica (resultado concreto: novo código), mensurável (100\% da força de trabalho deve ser treinada; avaliação mínima de 80\% de compreensão), alcançável (baixa complexidade operacional), relevante (atua como bússola institucional) e temporal (entrega definida). Seus KPIs — adesão ao treinamento, compreensão medida por avaliação e revisão anual obrigatória — convertem o código em instrumento vivo, acompanhado e atualizado, evitando que se transforme em norma decorativa.

O Programa Ética no Cuidado amplia o eixo formativo e é uma das ações mais profundas do plano, pois atua diretamente nas competências relacionais, comunicacionais e morais dos profissionais. O What é a criação de um programa permanente de formação ética, que não se limite a palestras esporádicas, mas integre módulos EAD, oficinas, simulações e estudos de caso. O Why é evidente no ambiente assistencial: decisões envolvendo consentimento informado, comunicação de más notícias, conflitos familiares, manejo de objeção de consciência e limites terapêuticos exigem habilidades técnicas, emocionais e éticas que não emergem espontaneamente. O Where — plataforma de educação permanente e salas de formação — assegura acessibilidade e continuidade. O When — meses quatro a doze — acompanha a maturação das demais ações, formando o tecido humano necessário para sustentar a mudança cultural. O Who inclui educadores internos, líderes assistenciais, equipe de educação permanente e representantes do comitê. O How envolve metodologias ativas, cenários simulados, videoaulas e estudos de caso reais. O How much — cerca de R\$ 25 mil anuais — é coerente com o porte da instituição e com a dimensão transformadora do programa. Em termos SMART, o programa é específico (conteúdos definidos), mensurável (meta de ≥ 85\% de cobertura e pós-testes ≥ 80\%), alcançável (recursos já existentes), relevante (impacto direto na qualidade do cuidado) e temporal (implantação do mês quatro ao doze). Os KPIs, especialmente taxa de adesão, desempenho nas avaliações e participação em oficinas, permitem verificar se o conhecimento está se convertendo em prática clínica mais segura e eticamente fundamentada.

A Política Institucional de Equidade no Atendimento incorpora diretamente os valores de justiça e igualdade material. O What é desenvolver fluxos de cuidado e protocolos de classificação que eliminem vieses conscientes e inconscientes. O Why reconhece que desigualdades sociais moldam profundamente o acesso e a experiência dos pacientes, e que instituições de saúde devem atuar como equalizadoras, não como reprodutoras dessas disparidades. O Where — ambulatório, pronto atendimento e internação — concentra áreas de maior impacto sobre acesso, acolhimento e fluxo. O When — a partir do quinto mês — permite que a política seja construída com base no mapeamento inicial de barreiras e injustiças percebidas. O Who inclui direção assistencial, enfermagem, serviço social e comunicação, garantindo olhar interdisciplinar. O How envolve revisão de fluxos, criação de materiais acessíveis, treinamento de equipes e monitoramento contínuo de eventuais disparidades. O How much — cerca de R\$ 40 mil — reflete investimentos em comunicação inclusiva e adequações de processos. Em relação ao SMART, a política é específica (criação de novos fluxos e materiais inclusivos), mensurável (tempo médio de atendimento por grupo social; queixas relacionadas à discriminação), alcançável (requer reorganização de práticas, não estruturas), altamente relevante (equidade é valor universal formador do ethos hospitalar) e temporal (implantação definida). Os KPIs — redução de queixas de discriminação em ≥ 20\% e diferença máxima de 10\% no tempo de atendimento entre grupos — fornecem métricas objetivas para avaliar se a instituição está avançando em justiça distributiva.

O Canal Ético e Espaço Seguro de Fala traduz honestidade e liberdade em mecanismos concretos de transparência. O What é desenvolver e implantar um canal que permita denúncias anônimas, registro de incidentes éticos e relato de situações de discriminação, abuso ou assédio. O Why se ancora na necessidade de criar um ambiente em que colaboradores e pacientes possam relatar problemas sem medo de retaliação, condição indispensável para organizações justas e saudáveis. O Where combina plataforma digital e atendimento telefônico, ampliando o acesso. O When — meses três a seis — garante que o canal esteja ativo logo após a criação do comitê e antes da maior parte das ações formativas. O Who envolve Compliance e TI, responsáveis por garantir sigilo, rastreabilidade e segurança da informação. O How centra-se na contratação de plataforma especializada e na criação de fluxos claros de apuração. O How much — cerca de R\$ 18 mil anuais — cobre a manutenção da plataforma. A ação atende plenamente ao SMART: é específica (criação do canal), mensurável (tempo médio de resposta, taxa de resolutividade), alcançável (tecnologia disponível), altamente relevante (proteção da integridade) e temporal (implantação definida). Seus KPIs — resposta inferior a sete dias úteis, resolutividade acima de 85\% e satisfação dos denunciantes — são medidores diretos do grau de confiança na instituição.

Os Protocolos de Decisão Ética representam o eixo clínico mais sofisticado, pois tratam das zonas cinzentas da prática médica. O What é produzir diretrizes que orientem decisões complexas relacionadas a fim de vida, escassez de recursos, conflitos familiares, objeções de consciência e limites terapêuticos. O Why é reduzir improvisos que geram insegurança moral, sofrimento das equipes e riscos legais. O Where envolve UTI, emergência e centro cirúrgico, locais onde dilemas aparecem com maior intensidade. O When — meses seis a doze — acompanha a ampliação da maturidade ética institucional. O Who inclui Comitê de Bioética, Direção Médica e Jurídico, assegurando fundamentação ética, técnica e normativa. O How inclui oficinas clínicas, simulações e ajustes baseados em casos reais. O How much — cerca de R\$ 10 mil — cobre horas de consultoria e elaboração dos documentos. O critério SMART é plenamente atendido: protocolos são específicos (com orientações claras), mensuráveis (taxa de adesão, redução de conflitos documentados), alcançáveis (baseados em práticas já validadas), relevantes (impacto direto na segurança) e temporais (implantação até o mês doze). Seus KPIs — adesão superior a 80\%, redução de conflitos e avaliação positiva pelas equipes — permitem observar se as diretrizes mudaram efetivamente a prática clínica.

O Relatório Anual de Ética, Inclusão e Sustentabilidade, publicado no décimo segundo mês, integra o conjunto de informações produzidas durante todo o plano. O What é consolidar indicadores éticos, dados de equidade, resolutividade do canal ético, participação em capacitações e desempenho dos protocolos. O Why se fundamenta na premissa de que transparência é condição essencial para credibilidade e governança. O Where envolve comunicação institucional e diretoria, que exercem papel-chave na divulgação e interpretação dos resultados. O When — mês doze — tem simbolismo de fechamento do ciclo. O Who inclui comitê, comunicação e controladoria, que combinam análise ética com rigor técnico. O How envolve coleta estruturada de dados, análise crítica e relato técnico. O How much — cerca de R\$ 12 mil — cobre diagramação e publicação. Em termos SMART, o relatório é específico (produto final concreto), mensurável (percentual de indicadores consolidados), alcançável (depende de dados já coletados), relevante (fundamental para governança) e temporal (publicação anual). KPIs como a consolidação de 95\% dos indicadores e a taxa de melhorias implementadas a partir do relatório demonstram se a instituição está fechando o ciclo de aprendizado e retroalimentando sua própria cultura ética.

Assim, ao longo do texto, percebe-se que cada ação foi desenhada não apenas como um item de um plano, mas como uma engrenagem interligada que articula valores, práticas e resultados concretos. O uso do 5W2H permite clareza e responsabilização, enquanto os critérios SMART garantem coerência e viabilidade. Os KPIs possibilitam monitoramento real e continuidade institucional, assegurando que os valores universalizáveis deixem de ser discursos genéricos e se tornem parâmetros objetivos para decisões cotidianas, relações profissionais e qualidade assistencial. O resultado esperado é a formação de uma cultura ética sólida, consciente e sustentável, capaz de enfrentar dilemas contemporâneos da saúde com maturidade institucional e compromisso humano.

\begin{center}
\includegraphics[width=0.47\textwidth]{fluxograma_valores_smart_kpi_colorido.png}
\end{center}

\section*{Cronograma – Plano de Ação (Gráfico de Gantt)}
\label{sec:orgeb22f31}

\begin{center}
\includegraphics[width=.9\linewidth]{gantt-plano-valores.png}
\end{center}
\end{document}