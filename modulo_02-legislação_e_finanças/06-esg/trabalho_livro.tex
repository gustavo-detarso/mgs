% Created 2025-11-07 sex 16:29
% Intended LaTeX compiler: xelatex
\documentclass[12pt]{article}
\usepackage{graphicx}
\usepackage{longtable}
\usepackage{wrapfig}
\usepackage{rotating}
\usepackage[normalem]{ulem}
\usepackage{amsmath}
\usepackage{amssymb}
\usepackage{capt-of}
\usepackage{hyperref}
\usepackage{fontspec}
\setmainfont{TeX Gyre Termes}[
UprightFont=texgyretermes-regular.otf,
BoldFont=texgyretermes-bold.otf,
ItalicFont=texgyretermes-italic.otf,
BoldItalicFont=texgyretermes-bolditalic.otf
]
\usepackage[brazil, ]{babel}
\usepackage{microtype}
\usepackage[left=3cm,right=2cm,top=3cm,bottom=2cm]{geometry}
\setlength{\parindent}{0pt}
\setlength{\parskip}{6pt}
\sloppy
\usepackage{xcolor}
\usepackage{hyperref}
\hypersetup{colorlinks=true, linkcolor=black, urlcolor=blue, citecolor=black}
\newcommand{\blueunder}[2][0.5pt]{%
{\vspace{-1.2cm}\centering\color{blue}\rule{#2}{#1}\par}%
}
\usepackage{tabularray}
\UseTblrLibrary{booktabs}
\SetTblrInner{rowsep=2pt,colsep=6pt}
\renewcommand{\arraystretch}{1.12}
\usepackage[most]{tcolorbox}
\tcbset{boxrule=0.6pt, arc=2mm, colframe=black!50, colback=white, left=6pt, right=6pt, top=6pt, bottom=6pt}
\author{Gustavo M. Mendes de Tarso}
\date{\today}
\title{}
\hypersetup{
 pdfauthor={Gustavo M. Mendes de Tarso},
 pdftitle={},
 pdfkeywords={},
 pdfsubject={},
 pdfcreator={Emacs 28.2 (Org mode 9.5.5)}, 
 pdflang={Pt_Br}}
\begin{document}

\begin{center}
\includegraphics[height=2cm]{/home/gustavodetarso/Documentos/.share/mgs_org/fgv.png}
\end{center}
\blueunder{0.5\linewidth}

{\centering\color{blue}\itshape MBA Gestão em Saúde\par}

\begin{tcolorbox}[title=Dados da obra,
  colback=gray!5,colframe=gray!40,boxrule=0.4pt,sharp corners]
\begin{tblr}{rows={t}, rowsep=6pt, colspec={Q[l,2.8cm] X[l]}}
\textbf{Autor(es)}   & Adriana Maria André \\
\textbf{Título}      & Gestão de Clínicas Hospitais e Indústrias da Saúde \\
\textbf{Edição}      & 3. ed. \\
\textbf{Publicação}  & Rio de Janeiro: Atheneu, 2022 \\
\textbf{Páginas}     & 362 p.; il.; 24 cm \\
\textbf{ISBN}        & 978-65-5586-540-0 \\
\textbf{Assuntos}    & Administração dos serviços de saúde; Administração da saúde \\
\textbf{Resumo}      & O capítulo defende que a gestão sustentável em saúde deve integrar ética, responsabilidade social e ambiental, conciliando lucro com valores humanos e coletivos. Propõe que clínicas e hospitais sejam economicamente viáveis, socialmente justos e ambientalmente responsáveis. \\
\end{tblr}
\end{tcolorbox}

\vspace{1cm}
\blueunder{0.5\linewidth}

ATIVIDADE 1 - Artigo Reflexivo sobre Leitura (1,0 ponto)

Vocês deverão ler o Capítulo 10 do livro "Gestão de Clínicas, Hospitais e Indústrias da Saúde" (3ª edição), coordenado pela Profa. Dra. Adriana Maria André (Editora Atheneu, 2023) - material disponível em anexo.

O que entregar:
\begin{itemize}
\item Um artigo reflexivo, elaborado por você, respondendo às seguintes questões:
\begin{itemize}
\item Quais as ideias principais defendidas pelo autor?
\item Qual delas mais chamou sua atenção? Por quê?
\item Que lição você leva da leitura deste capítulo para sua vida nas dimensões pessoal e profissional?
\end{itemize}
\end{itemize}

Requisitos:
\begin{itemize}
\item Extensão mínima: 500 palavras (aproximadamente 1 página)
\item Este é um trabalho de reflexão pessoal - queremos conhecer sua visão e análise crítica do conteúdo
\item Evite apenas resumir o capítulo; analise, relacione com sua experiência e expresse seu posicionamento
\end{itemize}

\vspace{1cm}
\blueunder{0.5\linewidth}

O capítulo se propõe a questionar o leitor, quanto a necessidade atual de deixar um negócio na área de saúde ser cada vez mais sustentável. Para tanto é construído um alicerce pautado em três pilares de sustentação: a ética, o social e o ambiental. Para se chegar a esta nova forma de capitalismo, foi necessário entender a evolução deste sistema econômico ao longo do tempo, que nos princípio, pautava-se exclusivamente na ideia de que o único sentido para uma empresa, seria a aquisição de lucro para os acionistas (\emph{shareholders}). Porém, esta visão foi suplantada atualmente para o modelo de \textbf{sustentabilidade}, que em síntese, redireciona a lucratividade para cobrir as necessidades de outros elementos da relação, chamados de \emph{stakeholders}, tão importantes quanto os acionistas. Sendo assim, o negócio deve apresentar um novo paradigma, o lucro ainda existe, mas ele deve agregar \textbf{valor}, estando intimamente conectado com os aspectos sociais e ambientais.   
\end{document}