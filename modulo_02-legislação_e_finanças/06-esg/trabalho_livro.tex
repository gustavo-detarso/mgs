% Created 2025-11-08 sáb 01:33
% Intended LaTeX compiler: lualatex
\documentclass[12pt]{article}
\usepackage{graphicx}
\usepackage{longtable}
\usepackage{wrapfig}
\usepackage{rotating}
\usepackage[normalem]{ulem}
\usepackage{amsmath}
\usepackage{amssymb}
\usepackage{capt-of}
\usepackage{hyperref}
\usepackage{fontspec}
\defaultfontfeatures{Ligatures=TeX}
\setmainfont{Times New Roman}[
Path = /usr/share/fonts/truetype/msttcorefonts/,
Extension = .ttf,
UprightFont = Times_New_Roman,
BoldFont = Times_New_Roman_Bold,
ItalicFont = Times_New_Roman_Italic,
BoldItalicFont = Times_New_Roman_Bold_Italic
]
\usepackage[brazil, ]{babel}
\usepackage[a4paper,margin=3cm]{geometry}
\linespread{1.5}
\setlength{\parskip}{0.7em}
\setlength{\parindent}{0pt}
\usepackage{xcolor}
\usepackage{hyperref}
\hypersetup{colorlinks=true, linkcolor=black, urlcolor=blue, citecolor=black}
\newcommand{\blueunder}[2][0.5pt]{%
{\vspace{-1.2cm}\centering\color{blue}\rule{#2}{#1}\par}%
}
\usepackage{tabularray}
\UseTblrLibrary{booktabs}
\SetTblrInner{rowsep=0pt,colsep=3pt}
\renewcommand{\arraystretch}{0.9}
\usepackage[most]{tcolorbox}
\tcbset{boxrule=0.6pt, arc=2mm, colframe=black!50, colback=white, left=3pt, right=2pt, top=3pt, bottom=2pt}
\author{Gustavo M. Mendes de Tarso}
\date{\today}
\title{}
\hypersetup{
 pdfauthor={Gustavo M. Mendes de Tarso},
 pdftitle={},
 pdfkeywords={},
 pdfsubject={},
 pdfcreator={Emacs 28.2 (Org mode 9.5.5)}, 
 pdflang={Pt_Br}}
\begin{document}

\begin{center}
\includegraphics[height=2cm]{/home/gustavodetarso/Documentos/.share/mgs_org/fgv.png}
\end{center}
\blueunder{0.5\linewidth}

{\centering\color{blue}\itshape MBA Gestão em Saúde\par}

\begingroup
\linespread{1}\selectfont
\begin{tcolorbox}[title=Dados da obra,
colback=gray!5,colframe=gray!40,boxrule=0.4pt,sharp corners]
\begin{tblr}[rowsep=1pt,stretch=1]{rows={t}, colspec={Q[l,2.8cm] X[l]}}
\textbf{Autor(es)}   & Adriana Maria André \\
\textbf{Título}      & Gestão de Clínicas, Hospitais e Indústrias da Saúde \\
\textbf{Edição}      & 3. ed. \\
\textbf{Publicação}  & Rio de Janeiro: Atheneu, 2022 \\
\textbf{Páginas}     & 362 p.; il.; 24 cm \\
\textbf{ISBN}        & 978-65-5586-540-0 \\
\textbf{Assuntos}    & 1. Administração dos serviços de saúde. 2. Administração da produção - Saúde \\
\textbf{Resumo}      & O capítulo defende que a gestão em saúde deve transcender o lucro e integrar ética, responsabilidade social e ambiental como pilares de sustentabilidade. Propõe um novo paradigma em que o desempenho das organizações seja medido pela coerência entre resultados econômicos e compromisso com a vida e o bem comum. \\
\end{tblr}
\end{tcolorbox}
\endgroup
\linespread{1.5}\selectfont

\vspace{1cm}
\blueunder{0.5\linewidth}

ATIVIDADE 1 - Artigo Reflexivo sobre Leitura (1,0 ponto)

Vocês deverão ler o Capítulo 10 do livro "Gestão de Clínicas, Hospitais e Indústrias da Saúde" (3ª edição), coordenado pela Profa. Dra. Adriana Maria André (Editora Atheneu, 2023) - material disponível em anexo.

O que entregar:
\begin{itemize}
\item Um artigo reflexivo, elaborado por você, respondendo às seguintes questões:
\begin{itemize}
\item Quais as ideias principais defendidas pelo autor?
\item Qual delas mais chamou sua atenção? Por quê?
\item Que lição você leva da leitura deste capítulo para sua vida nas dimensões pessoal e profissional?
\end{itemize}
\end{itemize}

Requisitos:
\begin{itemize}
\item Extensão mínima: 500 palavras (aproximadamente 1 página)
\item Este é um trabalho de reflexão pessoal - queremos conhecer sua visão e análise crítica do conteúdo
\item Evite apenas resumir o capítulo; analise, relacione com sua experiência e expresse seu posicionamento
\end{itemize}

\vspace{1cm}
\blueunder{0.5\linewidth}

O capítulo propõe uma reflexão profunda sobre o papel das organizações de saúde diante dos desafios contemporâneos da gestão. Ao discutir a transição do modelo econômico tradicional, baseado na maximização do lucro, para um paradigma de sustentabilidade, o texto evidencia que o setor da saúde deve ser pioneiro na consolidação de um novo modo de pensar o capitalismo. Essa mudança exige um alicerce firme, sustentado por três pilares: o ético, o social e o ambiental. O que o autor defende, em essência, é que a gestão em saúde precisa ser coerente com o seu propósito maior — preservar e promover a vida —, e isso só é possível quando a viabilidade econômica se equilibra com o compromisso moral e com o bem comum.

A ideia central do texto é clara: não existe gestão sustentável sem ética. Esse é o ponto de partida e o destino de toda a discussão. Considero, no entanto, paradoxal que um princípio tão elementar precise ser reafirmado. A ética, que deveria ser o pilar estrutural mais forte de qualquer sociedade civilizada, passou a ser tratada como uma tendência gerencial ou uma vantagem competitiva. Isso, por si só, é sintoma de uma deformação cultural. É quase um absurdo que, no século XXI, ainda seja necessário “lembrar” que um hospital ou uma clínica — espaços que lidam com o sofrimento humano — precisam ser éticos em suas decisões, relações e estratégias. Não há tecnologia, eficiência ou inovação capazes de compensar a ausência de integridade moral.

A parte que mais me chamou atenção foi justamente essa tentativa de reposicionar o lucro dentro de um contexto mais humano. O texto mostra que a lógica do shareholder, centrada no interesse exclusivo dos acionistas, foi sendo substituída por uma visão de stakeholder, que reconhece a importância de todos os atores envolvidos no processo produtivo e social. Embora essa mudança pareça positiva, ainda a vejo com certa desconfiança. Não porque discordo da ideia, mas porque me parece que o discurso da sustentabilidade, muitas vezes, é apropriado por empresas que apenas adaptam sua linguagem sem transformar sua essência. O que deveria ser um princípio ético acaba se tornando uma estratégia de imagem. Assim, a ética deixa de ser uma convicção e passa a ser uma conveniência.

No aspecto social, o capítulo traz uma reflexão especialmente relevante para quem atua na gestão pública ou privada da saúde. A responsabilidade social não se resume a cumprir leis trabalhistas ou apoiar campanhas filantrópicas. Ela exige um olhar sistêmico sobre o impacto das decisões administrativas. Uma clínica que valoriza seus profissionais, que promove condições dignas de trabalho e que respeita a individualidade de cada paciente já está exercendo sua função social. Da mesma forma, um gestor público que distribui recursos de forma equitativa, garantindo acesso à saúde de qualidade, também pratica responsabilidade social. O social, portanto, é o campo da ética em movimento — é onde as convicções morais se transformam em justiça concreta.

O aspecto ambiental complementa esse tripé de forma incontornável. Em um tempo de crise climática, falar de saúde sem falar de meio ambiente é uma incoerência. O autor lembra que hospitais, laboratórios e indústrias da saúde também impactam o ecossistema: geram resíduos, consomem energia e emitem gases nocivos. Gerir uma instituição de saúde é, portanto, também gerir seu impacto sobre o planeta. A responsabilidade ambiental, nesse sentido, não deve ser vista como custo, mas como investimento no futuro da própria humanidade. Preservar o meio ambiente é um ato de coerência ética: cuidar do planeta é cuidar da saúde coletiva.

Pessoalmente, o texto me levou a uma reflexão incômoda, mas necessária. A constatação de que a ética precisa ser “resgatada” revela o quanto nos acostumamos a viver em um modelo de mundo que normalizou a indiferença. Como indivíduo que não acredita no capitalismo como caminho de plenitude, mas que compreende a inevitabilidade de conviver com ele, percebo que a saída não está na negação do sistema, e sim na sua resignificação. Se o lucro é inevitável, que ele ao menos seja consequência de ações justas, humanas e equilibradas.

Na dimensão profissional, levo deste capítulo um lembrete valioso: liderar é um ato ético antes de ser um ato técnico. A gestão pública e a gestão em saúde não podem se reduzir a métricas ou resultados numéricos; elas precisam ser expressão de valores. Cada decisão administrativa, cada política implementada, cada escolha orçamentária deve carregar a responsabilidade de quem entende que o serviço prestado é um reflexo da própria humanidade de quem o executa.

Em síntese, o texto reforça uma ideia que considero inegociável: ética, responsabilidade social e compromisso ambiental não são modismos administrativos — são condições de legitimidade. Uma empresa ou instituição que ainda precise “adotar” esses valores demonstra que perdeu o sentido de sua própria existência. A verdadeira inovação não está em descobrir a ética, mas em finalmente praticá-la. A sustentabilidade, nesse contexto, não é um diferencial; é o mínimo que se espera de qualquer organização que pretenda existir de forma digna em um mundo que precisa, mais do que nunca, reaprender a ter consciência.
\section*{Estatísticas do texto}
\label{sec:org699eec6}
\begin{itemize}
\item Palavras (Org): 842
\item Linhas (PDF, FGV): 126
\item Parágrafos (Org): 9
\end{itemize}
\end{document}