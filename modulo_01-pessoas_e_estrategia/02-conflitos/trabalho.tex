% Created 2025-02-23 dom 19:32
% Intended LaTeX compiler: pdflatex
\documentclass[article,12pt,oneside,a4paper]{article}
\usepackage[utf8]{inputenc}
\usepackage[T1]{fontenc}
\usepackage{graphicx}
\usepackage{longtable}
\usepackage{wrapfig}
\usepackage{rotating}
\usepackage[normalem]{ulem}
\usepackage{amsmath}
\usepackage{amssymb}
\usepackage{capt-of}
\usepackage{hyperref}
\usepackage[brazil, brazilian]{babel}
\usepackage[utf8]{inputenc}
\usepackage[T1]{fontenc}
\usepackage{lmodern}
\usepackage{microtype}
\usepackage[left=3cm,top=3cm,right=2cm,bottom=2cm]{geometry}
\frenchspacing
\pretolerance=100
\tolerance=200
\emergencystretch=3em
\hbadness=2000
\usepackage[alf]{abntex2cite}
\bibliographystyle{abntex2-alf}
\author{Gustavo Magalhães Mendes de Tarso}
\date{\today}
\title{Análise de um Caso de Negociação e Gestão de Conflito}
\hypersetup{
 pdfauthor={Gustavo Magalhães Mendes de Tarso},
 pdftitle={Análise de um Caso de Negociação e Gestão de Conflito},
 pdfkeywords={},
 pdfsubject={},
 pdfcreator={Emacs 30.0.91 (Org mode 9.7.11)}, 
 pdflang={Portuges}}
\begin{document}

\maketitle
\section{RESUMO}
\label{sec:orgcdce6e7}

A negociação é uma habilidade essencial no ambiente corporativo, permitindo alinhar interesses divergentes e resolver conflitos de forma eficaz. Este trabalho analisa um caso real de negociação ocorrido no Ministério da Previdência Social, em que identifiquei um erro matemático na distribuição de vagas do concurso de Perícia Médica Federal. O embate ocorreu com minha superior hierárquica, que resistiu à correção do erro, sustentando sua decisão com base em autoridade hierárquica, apesar das evidências técnicas apresentadas. A análise destaca a importância da escuta ativa e da paciência como fatores críticos para a resolução de conflitos e a efetividade na negociação. O estudo aprofunda conceitos de negociação e gestão de conflitos, discutindo estratégias que podem aprimorar a comunicação interpessoal e minimizar desgastes em processos decisórios. \\

Palavras-chave: negociação, gestão de conflitos, escuta ativa, paciência.
\section{INTRODUÇÃO}
\label{sec:orgd5a2473}

A negociação está presente em diversas esferas da vida profissional e organizacional, sendo um instrumento essencial para a resolução de conflitos e a tomada de decisões eficazes \cite{fisher1991getting}. No contexto corporativo, lidar com divergências de opinião e interesses exige habilidades que vão além do conhecimento técnico, incluindo comunicação estratégica, inteligência emocional e gestão de conflitos \cite{goleman1998trabalhando}.

Neste trabalho, relato minha experiência em um conflito ocorrido no setor público, no qual, como servidor técnico responsável pela análise matemática da distribuição de vagas do concurso de Perícia Médica Federal, identifiquei uma inconsistência na metodologia adotada. A falha, se mantida, comprometeria a equidade do processo e poderia resultar em questionamentos administrativos e jurídicos. Ao apresentar a inconsistência e sugerir correções embasadas em critérios técnicos objetivos, enfrentei resistência da minha superior hierárquica, que manteve sua posição com base na autoridade do cargo. A experiência evidenciou a necessidade de aprimoramento em duas competências fundamentais para a negociação: escuta ativa e paciência \cite{ury1993como}.
\section{REFERENCIAL TEÓRICO}
\label{sec:orgaed2701}
\subsection{Negociação e Estilos de Resolução de Conflitos}
\label{sec:org9bb1d91}

A negociação é um processo de comunicação estruturado, cujo objetivo é resolver conflitos, alinhar interesses e alcançar um acordo mutuamente satisfatório \cite{lewicki2015negotiation}. No ambiente organizacional, a negociação desempenha um papel essencial na tomada de decisões e na gestão de conflitos, sendo frequentemente influenciada por fatores como poder hierárquico, cultura organizacional e a disposição das partes em cooperar ou competir.

Segundo Robbins e Judge \cite{robbins2015comportamento}, os estilos de negociação podem ser classificados em cinco categorias principais. Cada estilo possui características distintas e sua eficácia depende do contexto e dos objetivos das partes envolvidas:
\subsubsection{Negociação Competitiva}
\label{sec:org17ebe99}

O estilo competitivo é caracterizado pelo foco na maximização dos próprios interesses, muitas vezes em detrimento das necessidades da outra parte. Negociadores competitivos tendem a adotar uma postura assertiva e utilizam estratégias de pressão para alcançar seus objetivos. \\

Características principais:

\begin{itemize}
\item Alta assertividade e baixa cooperação.
\item Ênfase na obtenção de ganhos individuais.
\item Uso de táticas de persuasão, influência e, em alguns casos, coerção.
\item Menor preocupação com a manutenção do relacionamento.
\end{itemize}

Quando usar?

\begin{itemize}
\item Quando há recursos limitados e é necessário obter a maior vantagem possível.
\item Em situações de negociação única, onde não há interesse na construção de um relacionamento de longo prazo.
\item Quando há um desequilíbrio de poder a favor do negociador e a outra parte tem pouca margem de manobra.
\end{itemize}

Riscos:

\begin{itemize}
\item Pode gerar resistência e animosidade, dificultando negociações futuras.
\item A outra parte pode reagir de maneira igualmente agressiva, tornando o processo um impasse.
\item Pode resultar em acordos instáveis ou prejudiciais ao longo prazo.
\end{itemize}
\subsubsection{Negociação Colaborativa}
\label{sec:orgda140d4}

O estilo colaborativo busca gerar valor para ambas as partes, enfatizando um resultado benéfico para todos os envolvidos. Esse estilo de negociação é pautado na transparência, no diálogo aberto e na busca por soluções criativas que atendam aos interesses de todos. \\

Características principais:

\begin{itemize}
\item Alta assertividade e alta cooperação.
\item Priorização da criação de valor e da satisfação mútua.
\item Compartilhamento de informações e interesses para encontrar soluções inovadoras.
\item Construção de relações de confiança e parcerias duradouras.
\end{itemize}

Quando usar?

\begin{itemize}
\item Quando há espaço para um acordo vantajoso para ambas as partes (ganha-ganha).
\item Em negociações estratégicas de longo prazo.
\item Quando há um problema complexo que exige soluções criativas e flexíveis.
\end{itemize}

Benefícios:

\begin{itemize}
\item Gera acordos sustentáveis e duradouros.
\item Fortalece relações de confiança e cooperação.
\item Reduz conflitos e melhora a comunicação organizacional.
\end{itemize}

Desafios:

\begin{itemize}
\item Requer mais tempo e esforço para chegar a um consenso.
\item Pode ser ineficaz se uma das partes adotar um comportamento competitivo.
\item Exige que ambas as partes estejam dispostas a compartilhar informações e trabalhar em conjunto.
\end{itemize}
\subsubsection{Negociação Acomodativa}
\label{sec:orgcd7769b}

O estilo acomodativo é utilizado quando um negociador decide ceder à outra parte para preservar o relacionamento ou evitar conflitos. Esse estilo pode ser útil em situações em que a relação interpessoal é mais valiosa do que o resultado imediato da negociação. \\

Características principais:

\begin{itemize}
\item Baixa assertividade e alta cooperação.
\item Priorização do relacionamento em detrimento dos próprios interesses.
\item Disposição para ceder para evitar impasses.
\end{itemize}

Quando usar?

\begin{itemize}
\item Quando a questão negociada tem baixa importância para o negociador, mas alta importância para a outra parte.
\item Quando a preservação do relacionamento é mais relevante do que o resultado específico da negociação.
\item Em situações em que há uma grande disparidade de poder e resistência pode trazer consequências negativas.
\end{itemize}

Riscos:

\begin{itemize}
\item Pode gerar frustração se o estilo for utilizado com frequência, levando a concessões excessivas.
\item Pode incentivar comportamentos exploratórios por parte do outro negociador.
\item Pode enfraquecer a posição do negociador a longo prazo.
\end{itemize}
\subsubsection{Negociação Evitativa}
\label{sec:org0719a16}

O estilo evitativo ocorre quando uma das partes opta por não se envolver diretamente na negociação, seja por considerar que os custos da negociação superam os benefícios ou por não ter um posicionamento claro sobre a questão. \\

Características principais:

\begin{itemize}
\item Baixa assertividade e baixa cooperação.
\item Recusa ou adiamento da negociação.
\item Ausência de confronto direto.
\end{itemize}

Quando usar?

\begin{itemize}
\item Quando o tema da negociação não é prioritário e não há necessidade de uma decisão imediata.
\item Quando os custos de entrar em um conflito são maiores do que os benefícios esperados.
\item Quando há informações insuficientes e é necessário mais tempo para avaliar a situação.
\end{itemize}

Riscos:

\begin{itemize}
\item Pode resultar na perda de oportunidades estratégicas.
\item Se utilizada excessivamente, pode prejudicar a credibilidade do negociador.
\item Pode levar a conflitos futuros, caso o problema não seja resolvido adequadamente.
\end{itemize}
\subsubsection{Negociação Compromissada}
\label{sec:orgf46e76e}

O estilo compromissado busca equilibrar interesses, promovendo concessões mútuas para alcançar um acordo razoável para ambas as partes. Esse estilo é frequentemente utilizado quando há impasses e as partes precisam encontrar um meio-termo. \\

Características principais:

\begin{itemize}
\item Nível moderado de assertividade e cooperação.
\item Enfoque na busca de soluções intermediárias.
\item Disposição para fazer e receber concessões.
\end{itemize}

Quando usar?

\begin{itemize}
\item Quando o tempo é um fator crítico e uma solução rápida é necessária.
\item Quando os interesses de ambas as partes são relativamente equilibrados.
\item Quando um acordo parcial é melhor do que nenhum acordo.
\end{itemize}

Benefícios:

\begin{itemize}
\item Agilidade na resolução de conflitos.
\item Pode evitar impasses prolongados.
\item Garante que ambas as partes obtenham algum benefício.
\end{itemize}

Desafios:

\begin{itemize}
\item Pode resultar em acordos subótimos, nos quais nenhuma das partes está totalmente satisfeita.
\item Pode ser explorado por negociadores mais experientes, que conseguem extrair concessões maiores.
\item Em algumas situações, pode enfraquecer a posição de negociação ao longo do tempo.
\end{itemize}

A escolha do estilo de negociação adequado depende do contexto e das motivações das partes envolvidas \cite{malhotra2016negotiation}. Enquanto o estilo competitivo pode ser útil em negociações pontuais, o colaborativo tende a gerar acordos mais sustentáveis. O acomodativo e o evitativo podem ser utilizados estrategicamente para preservar relacionamentos, enquanto o compromissado é eficiente para situações em que um consenso rápido é necessário.

Para um negociador eficaz, é essencial compreender a dinâmica de cada abordagem e desenvolver flexibilidade para adaptar-se conforme o contexto. O uso estratégico dos estilos de negociação permite não apenas alcançar melhores acordos, mas também fortalecer relacionamentos profissionais e aprimorar a comunicação interpessoal.
\subsection{Poder e Autoridade na Gestão de Conflitos}
\label{sec:orgc96864f}

Em ambientes hierárquicos, o poder organizacional influencia diretamente as negociações. French e Raven \cite{french1959bases} identificam diferentes tipos de poder:

\begin{itemize}
\item \emph{Poder legítimo}: Deriva da posição formal.
\item \emph{Poder de especialização}: Baseado no conhecimento técnico.
\item \emph{Poder coercitivo}: Capacidade de impor sanções.
\item \emph{Poder de recompensa}: Controle sobre benefícios.
\end{itemize}

No caso analisado, minha superior hierárquica utilizou seu poder legítimo para sustentar sua posição, enquanto eu me baseei no poder de especialização para justificar minha argumentação e demonstrar tecnicamente a necessidade da revisão do cálculo.
\subsection{Retórica Aristotélica e Persuasão na Negociação}
\label{sec:orgaf8294f}

Aristóteles descreve três pilares essenciais para a persuasão \cite{aristoteles2004retorica}:

\begin{itemize}
\item \textbf{Logos} – Argumentação baseada em lógica e evidências.
\item \textbf{Ethos} – Construção de credibilidade do orador.
\item \textbf{Pathos} – Apelo emocional para gerar conexão.
\end{itemize}

O sucesso da negociação depende do equilíbrio entre esses elementos.
\section{ANÁLISE DO CASO}
\label{sec:org01344e9}
\subsection{O Conflito e Seus Desdobramentos}
\label{sec:org589a516}

No setor responsável pela elaboração e distribuição de vagas do concurso de Perícia Médica Federal, identifiquei um erro matemático na metodologia adotada. Caso aplicado, esse cálculo distorceria a alocação das vagas, comprometendo a eficiência do processo e gerando risco de questionamentos jurídicos.

Ao apresentar a inconsistência e propor ajustes baseados em critérios matemáticos objetivos, minha superior hierárquica recusou a revisão, utilizando sua autoridade hierárquica para validar a decisão inicial. Essa postura centralizadora e resistente à argumentação técnica impôs um desafio à negociação, tornando essencial o uso de estratégias de persuasão para evitar que o erro fosse mantido.

A manutenção da metodologia incorreta, apesar das evidências técnicas apresentadas, poderia resultar em impactos significativos para a validade do concurso. Se a distribuição de vagas ocorresse com base em um critério equivocado, haveria risco de impugnação do processo seletivo, comprometendo sua credibilidade e demandando retificações posteriores. Além disso, tal erro comprometeria a eficiência da gestão pública e a alocação racional de recursos.

Diante desse cenário, enfatizei a necessidade de revisão da metodologia, garantindo que a distribuição de vagas fosse baseada em parâmetros matemáticos objetivos e alinhados aos princípios de transparência, isonomia e legalidade. A adoção de um critério técnico adequado era essencial para assegurar que a seleção atendesse aos interesses da administração pública e dos candidatos de maneira justa e eficiente.
\subsection{Estratégias Utilizadas e aplicação do Raciocínio Aristotélico}
\label{sec:orgf0a26fc}

Para superar a resistência da chefe e evitar a escalada do conflito, foram adotadas as seguintes estratégias:

\begin{itemize}
\item \textbf{Logos (Razão e Evidências)}:  
\begin{itemize}
\item Apresentação de dados técnicos e simulações para demonstrar a falha no cálculo.
\item Comparação com metodologias validadas para reforçar a argumentação.
\end{itemize}

\item \textbf{Ethos (Credibilidade e Autoridade Técnica)}:  
\begin{itemize}
\item Citação de experiências anteriores do servidor na área.
\item Referências a normativas internas e padrões técnicos.
\end{itemize}

\item \textbf{Pathos (Empatia e Comunicação Estratégica)}:  
\begin{itemize}
\item Uso de perguntas abertas para envolver a chefe na análise.
\item Comunicação respeitosa para evitar resistência emocional.
\end{itemize}
\end{itemize}

Essas abordagens permitiram que a chefe reconsiderasse sua posição sem sentir que sua autoridade estava sendo diretamente questionada.
\subsection{Alternativas das Partes (MACNA)}
\label{sec:org6bb7644}

Segundo Fisher e Ury \cite{fisher1991getting}, a \textbf{Melhor Alternativa em Caso de Não Acordo (MACNA)} deve ser considerada para avaliar a força da posição de cada parte.

\begin{itemize}
\item \textbf{Minha MACNA}:  
\begin{itemize}
\item Escalar a questão para instâncias superiores (Tribunal de Contas, auditorias internas).
\item Emitir um parecer técnico formalizando a contestação.
\end{itemize}

\item \textbf{MACNA da chefe}:  
\begin{itemize}
\item Imposição unilateral da decisão, assumindo riscos administrativos.
\item Consulta a outro especialista para validar sua posição.
\end{itemize}
\end{itemize}

A análise indicava que a MACNA da chefe era mais fraca, favorecendo uma negociação bem conduzida.
\subsection{CONEXÃO COM O ARTIGO \emph{HOW TO DISAGREE WITH SOMEONE MORE POWERFUL THAN YOU}}
\label{sec:orgd4193c2}

O caso analisado neste trabalho possui forte relação com as diretrizes apresentadas no artigo de Amy Gallo \cite{gallo2016how}, que aborda estratégias para discordar de superiores hierárquicos de forma respeitosa e produtiva. No artigo, Gallo sugere técnicas que se alinham diretamente às abordagens utilizadas pelo servidor técnico na negociação com sua chefe.
\subsubsection{Avaliação de Riscos ao Discordar}
\label{sec:org5fe1a0b}

Gallo destaca que é natural temer as consequências de discordar de um superior, mas que é importante ponderar os riscos de não falar nada. No caso analisado, a manutenção do erro matemático poderia comprometer a eficiência da distribuição de vagas e gerar questionamentos administrativos e jurídicos.
\subsubsection{Construção de um Argumento Estratégico}
\label{sec:org12a410a}

O artigo sugere que a discordância deve ser estruturada com base na identificação de um objetivo comum e na apresentação de evidências lógicas (Logos). No caso estudado, o servidor utilizou essa abordagem ao apresentar simulações matemáticas e normativas internas para embasar sua argumentação.
\subsubsection{Técnicas de Comunicação}
\label{sec:org15da8eb}

Gallo recomenda evitar linguagem julgadora e focar apenas nos fatos. O servidor técnico seguiu essa orientação ao apresentar seu argumento de forma técnica, sem ataques pessoais.

A estrutura \emph{Logos}, \emph{Ethos}, \emph{Pathos} também aparece como um elemento central tanto no artigo de Gallo quanto no estudo do caso:

\begin{itemize}
\item \emph{Logos}: Apresentação de dados concretos e técnicos.

\item \emph{Ethos}: Reforço da credibilidade profissional do servidor.

\item \emph{Pathos}: Uso de comunicação respeitosa para reduzir resistência emocional.
\end{itemize}
\subsubsection{Validação do Ponto de Vista do Superior}
\label{sec:orga845694}

Uma das estratégias sugeridas no artigo é reconhecer a perspectiva do superior antes de apresentar uma objeção. No caso estudado, o servidor poderia ter iniciado destacando a lógica por trás do cálculo original antes de sugerir ajustes, minimizando a percepção de confronto.

A integração dessas abordagens fortalece a análise do caso, evidenciando como a negociação bem conduzida pode transformar conflitos em oportunidades de melhoria organizacional.
\section{CONCLUSÃO}
\label{sec:orgf5d5404}
A análise detalhada deste caso evidencia que negociações em ambientes hierárquicos exigem a combinação de \textbf{habilidade técnica e inteligência interpessoal}. Ao identificar o erro e enfrentar a resistência de minha superior hierárquica, foi necessário estruturar um argumento técnico sólido, equilibrando escuta ativa, paciência e persuasão estratégica. A adoção de uma abordagem racional baseada na retórica aristotélica permitiu que a revisão do cálculo fosse realizada sem comprometer a dinâmica organizacional.

Minha experiência reforça a importância do desenvolvimento contínuo de habilidades de negociação para lidar com conflitos de maneira eficaz e construtiva, garantindo que a tomada de decisão esteja alinhada aos princípios de transparência e eficiência administrativa.
\renewcommand{\bibname}{REFERÊNCIAS}
\bibliography{referencias}
\end{document}
